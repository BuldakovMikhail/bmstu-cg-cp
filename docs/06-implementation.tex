\chapter{Технологический раздел}

В данном разделе будут представлены средства разработки программного обеспечения, детали реализации и процесс сборки разрабатываемого
программного обеспечения.

\section{Выбор средств реализации}

В качестве языка программирования для разработки программного обеспечения был выбран язык \textit{GLSL} \cite{glsl}. Данный выбор обусловлен следующим:
\begin{itemize}
	\item поддержка языком всех структур данных, выбранных в результате проектирования;
	\item возможность реализовать все алгоритмы, выбранные в результате проектирования.
\end{itemize}


Для создания графического интерфейса был выбран язык \textit{Python} \cite{python}, поскольку данный язык позволяет работать с библиотекой \textit{ModernGL} \cite{moderngl}.
 
Графический интерфейс создан с помощью библиотеки \textit{PyQt} \cite{pyqt}. Данная библиотека содержит в себе объекты, позволяющие интегрировать \textit{OpenGL} контекст, а также возможности создания панели управления с кнопками, что позволит в интерактивном режиме управлять изображением.

В качестве среды разработки был выбран текстовый редактор Visual Studio Code \cite{vsc}, поддерживающий возможность установки плагинов и обладающий системой отладки кода.


\section{Реализация алгоритмов}

Листинги реализаций алгоритмов находятся в приложении \ref{app:listings}.

На листинге \ref{lst:density.glsl} представлена реализация алгоритма вычисления плотности в точке пространства. На листинге \ref{lst:lightenergy.glsl} представлена реализация алгоритма вычисления освещенности в точке пространства. На листинге \ref{lst:raymarching.glsl} представлена реализация алгоритма Ray Marching.


\section{Результаты разработки}

При запуске программы на сцене отображаются облака с предустановленными параметрами. Для изменения параметров облаков справой стороны окна имеются поля ввода. Интерфейс разработанного ПО изображен на рисунке \ref{img:interface}.


\includeimage
{interface} % Имя файла без расширения (файл должен быть расположен в директории inc/img/)
{f} % Обтекание (без обтекания)
{h} % Положение рисунка (см. figure из пакета float)
{1\textwidth} % Ширина рисунка
{Интерфейс разработанного ПО} % Подпись рисунка


На рисунке \ref{img:silverlining} изображена визуализация облаков в разработанном ПО на просвете солнца. На полученном изображении видна яркая окантовка облака и более темная сердцевина, такой же эффект наблюдается в реальных облаках. Для сравнения на рисунке \ref{img:silverliningreal} представлена фотография реального облака на солнечном просвете.

\includeimage
{silverlining} % Имя файла без расширения (файл должен быть расположен в директории inc/img/)
{f} % Обтекание (без обтекания)
{h} % Положение рисунка (см. figure из пакета float)
{0.8\textwidth} % Ширина рисунка
{Визаулиазцаия облаков на солнечном просвете} % Подпись рисунка


\includeimage
{silverliningreal} % Имя файла без расширения (файл должен быть расположен в директории inc/img/)
{f} % Обтекание (без обтекания)
{h} % Положение рисунка (см. figure из пакета float)
{0.8\textwidth} % Ширина рисунка
{Фотография облаков на солнечном просвете} % Подпись рисунка

На рисунке \ref{img:sugarpow} показан результат визуализации неба для случая, когда угол между падающим солнечным лучом и лучом наблюдения близок к 180 градусам. На полученном изображение видны неровности на поверхности облаков. 

\includeimage
{sugarpow} % Имя файла без расширения (файл должен быть расположен в директории inc/img/)
{f} % Обтекание (без обтекания)
{h} % Положение рисунка (см. figure из пакета float)
{0.8\textwidth} % Ширина рисунка
{Визаулизация облаков, в случае когда угол между падающим солнечным лучом и лучом наблюдения близок к 180 градусам} % Подпись рисунка


\section{Функциональное тестирование}

\textbf{Тест 1: установка максимального покрытия.}

Ожидаемый результат: облака покрывают все небо.

Фактический результат изображен на рисунке \ref{img:test1}. Тест пройден.

\includeimage
{test1} % Имя файла без расширения (файл должен быть расположен в директории inc/img/)
{f} % Обтекание (без обтекания)
{h} % Положение рисунка (см. figure из пакета float)
{0.6\textwidth} % Ширина рисунка
{Результат теста 1. Небо полностью покрыто облаками} % Подпись рисунка


\textbf{Тест 2: установка минимального покрытия.}

Ожидаемый результат: на небе нет облаков.

Фактический результат изображен на рисунке \ref{img:test2}. Тест пройден.

\includeimage
{test2} % Имя файла без расширения (файл должен быть расположен в директории inc/img/)
{f} % Обтекание (без обтекания)
{h} % Положение рисунка (см. figure из пакета float)
{0.6\textwidth} % Ширина рисунка
{Результат теста 2. На небе нет облаков} % Подпись рисунка





\clearpage
\section{Вывод}

В данном разделе были представлены средства разработки программного обеспечения и детали реализации. В итоге был получен рендер следующего кадра:


