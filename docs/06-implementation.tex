\chapter{Технологический раздел}

В данном разделе будут представлены средства разработки программного обеспечения, детали реализации и процесс сборки разрабатываемого программного обеспечения.

\section{Выбор средств реализации}

В качестве языка программирования для разработки программного обеспечения был выбран язык \textit{GLSL}~\cite{glsl}. Данный выбор обусловлен следующим:
\begin{itemize}
	\item поддержка языком всех структур данных, выбранных в результате проектирования;
	\item возможность реализовать все алгоритмы, выбранные в результате проектирования.
\end{itemize}


Для создания графического интерфейса был выбран язык \textit{Python}~\cite{python}, поскольку данный язык позволяет работать с библиотекой \textit{ModernGL}~\cite{moderngl}.
 
Графический интерфейс создан с помощью библиотеки \textit{PyQt}~\cite{pyqt}. Данная библиотека содержит в себе объекты, позволяющие интегрировать \textit{OpenGL} контекст, а также возможности создания панели управления с кнопками, что позволит в интерактивном режиме управлять изображением.

В качестве среды разработки был выбран текстовый редактор Visual Studio Code~\cite{vsc}, поддерживающий возможность установки плагинов и обладающий системой отладки кода.


\section{Реализация алгоритмов}

Листинги реализаций алгоритмов находятся в приложении~\ref{app:listings}.

На листинге~\ref{lst:density.glsl} представлена реализация алгоритма вычисления плотности в точке пространства. На листинге~\ref{lst:lightenergy.glsl} представлена реализация алгоритма вычисления освещенности в точке пространства. На листинге~\ref{lst:raymarching.glsl} представлена реализация алгоритма Ray Marching.


\section{Результаты разработки}

При запуске программы на сцене отображаются облака с предустановленными параметрами. Для изменения параметров облаков с правой стороны окна имеются поля ввода. Интерфейс разработанного программного обеспечения изображен на рисунке~\ref{img:interface}.


\includeimage
{interface} % Имя файла без расширения (файл должен быть расположен в директории inc/img/)
{f} % Обтекание (без обтекания)
{H} % Положение рисунка (см. figure из пакета float)
{1\textwidth} % Ширина рисунка
{Интерфейс разработанного программного обеспечения} % Подпись рисунка


На рисунке~\ref{img:silverlining} изображена визуализация облаков в разработанном программном обеспечении с просветами солнца. На полученном изображении видна яркая граница облака и более темная сердцевина, такой же эффект наблюдается в реальных облаках. Для сравнения на рисунке~\ref{img:silverliningreal} представлена фотография реального облака с солнечным просветом.

\includeimage
{silverlining} % Имя файла без расширения (файл должен быть расположен в директории inc/img/)
{f} % Обтекание (без обтекания)
{H} % Положение рисунка (см. figure из пакета float)
{0.8\textwidth} % Ширина рисунка
{Визуализация облаков с солнечным просветом} % Подпись рисунка


\includeimage
{silverliningreal} % Имя файла без расширения (файл должен быть расположен в директории inc/img/)
{f} % Обтекание (без обтекания)
{H} % Положение рисунка (см. figure из пакета float)
{0.8\textwidth} % Ширина рисунка
{Фотография облаков с солнечным просветом} % Подпись рисунка

На рисунке~\ref{img:sugarpow} показан результат визуализации неба для случая, когда падающий солнечный луч и луч наблюдения практически сонаправлены. На полученном изображении видны неровности на поверхности облаков. 

\includeimage
{sugarpow} % Имя файла без расширения (файл должен быть расположен в директории inc/img/)
{f} % Обтекание (без обтекания)
{H} % Положение рисунка (см. figure из пакета float)
{0.8\textwidth} % Ширина рисунка
{Визуализация, демонстрирующая неровную поверхность облаков} % Подпись рисунка

\clearpage

\section{Функциональное тестирование}

\textbf{Тест 1: установка максимального покрытия.}

Ожидаемый результат: облака покрывают все небо.

Фактический результат изображен на рисунке~\ref{img:test1}. Тест пройден.

\includeimage
{test1} % Имя файла без расширения (файл должен быть расположен в директории inc/img/)
{f} % Обтекание (без обтекания)
{H} % Положение рисунка (см. figure из пакета float)
{0.6\textwidth} % Ширина рисунка
{Результат теста 1. Небо полностью покрыто облаками} % Подпись рисунка


\textbf{Тест 2: установка минимального покрытия.}

Ожидаемый результат: на небе нет облаков.

Фактический результат изображен на рисунке~\ref{img:test2}. Тест пройден.

\includeimage
{test2} % Имя файла без расширения (файл должен быть расположен в директории inc/img/)
{f} % Обтекание (без обтекания)
{H} % Положение рисунка (см. figure из пакета float)
{0.6\textwidth} % Ширина рисунка
{Результат теста 2. На небе нет облаков} % Подпись рисунка


\textbf{Тест 3: установка плотности 0.}

Ожидаемый результат: на небе нет облаков.

Фактический результат изображен на рисунке~\ref{img:test3}. Тест пройден.

\includeimage
{test3} % Имя файла без расширения (файл должен быть расположен в директории inc/img/)
{f} % Обтекание (без обтекания)
{H} % Положение рисунка (см. figure из пакета float)
{0.6\textwidth} % Ширина рисунка
{Результат теста 3. На небе нет облаков} % Подпись рисунка



\textbf{Тест 4: установка малой плотности.}

Ожидаемый результат: у облаков практически отсутствует тень.

Фактический результат изображен на рисунке~\ref{img:test4}. Тест пройден.

\includeimage
{test4} % Имя файла без расширения (файл должен быть расположен в директории inc/img/)
{f} % Обтекание (без обтекания)
{H} % Положение рисунка (см. figure из пакета float)
{0.6\textwidth} % Ширина рисунка
{Результат теста 4. Облака малой плотности} % Подпись рисунка


\textbf{Тест 5: установка максимальной плотности.}

Ожидаемый результат: у облаков ярка выражена тень.

Фактический результат изображен на рисунке~\ref{img:test4}. Тест пройден.

\includeimage
{test5} % Имя файла без расширения (файл должен быть расположен в директории inc/img/)
{f} % Обтекание (без обтекания)
{H} % Положение рисунка (см. figure из пакета float)
{0.6\textwidth} % Ширина рисунка
{Результат теста 5. Облака максимальной плотности} % Подпись рисунка


\textbf{Тест 6: установка вектора падения света в 0.}

Ожидаемый результат: сообщение об ошибке.

Фактический результат: сообщение об ошибке. Тест пройден.


\textbf{Тест 7: установка вектора взгляда в 0.}

Ожидаемый результат: сообщение об ошибке.

Фактический результат: сообщение об ошибке. Тест пройден.


\section*{Вывод}

В данном разделе были представлены средства разработки программного обеспечения и детали реализации. В итоге было разработано программное обеспечение, позволяющее визуализировать облачное небо.


