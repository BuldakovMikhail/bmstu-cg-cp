\chapter{Технологический раздел}

В данном разделе будут представлены средства разработки программного обеспечения, детали реализации и процесс сборки разрабатываемого
программного обеспечения.

\section{Выбор средств реализации}

В качестве языка программирования для разработки программного обеспечения был выбран язык \textit{GLSL} \cite{glsl}. Данный выбор обусловлен следующим:
\begin{itemize}
	\item поддержка языком всех структур данных, выбранных в результате проектирования;
	\item возможность реализовать все алгоритмы, выбранные в результате проектирования.
\end{itemize}


Для создания окна был выбран язык \textit{Python} \cite{python}, поскольку данный язык позволяет работать с библиотекой \textit{ModernGL} \cite{moderngl}.


\section{Реализация алгоритмов}

На листингах \ref{lst:density} -- \ref{lst:lightmodel}, приведенных ниже, описаны основные алгоритмы используемые для визуализации облаков на языке шейдеров GLSL.



\clearpage

\includelistingpretty
{density.glsl} % Имя файла с расширением (файл должен быть расположен в директории inc/lst/)
{} % Язык программирования (необязательный аргумент)
{Реализация функции определяющей плотность в точке} % Подпись листинга


\clearpage

\includelistingpretty
{raymarching.glsl} % Имя файла с расширением (файл должен быть расположен в директории inc/lst/)
{} % Язык программирования (необязательный аргумент)
{Реализация алгоритма Ray Marching} % Подпись листинга


\clearpage

\includelistingpretty
{lightenergy.glsl} % Имя файла с расширением (файл должен быть расположен в директории inc/lst/)
{} % Язык программирования (необязательный аргумент)
{Реализация модели освещения} % Подпись листинга





\section{Вывод}

В данном разделе были представлены средства разработки программного обеспечения и детали реализации. В итоге был получен рендер следующего кадра:

\includeimage
{screen} % Имя файла без расширения (файл должен быть расположен в директории inc/img/)
{f} % Обтекание (без обтекания)
{h} % Положение рисунка (см. figure из пакета float)
{0.6\textwidth} % Ширина рисунка
{Кадр, облачного неба, созданный с помощью описанных алгоритмов} % Подпись рисунка

