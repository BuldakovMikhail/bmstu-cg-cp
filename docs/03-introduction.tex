\chapter*{ВВЕДЕНИЕ}
\addcontentsline{toc}{chapter}{ВВЕДЕНИЕ}

Компьютерная графика представляет собой совокупность методов
и способов преобразования информации в графическое представление при
помощи ЭВМ.

Наличие облаков сильно влияет на восприятие изображения, делая его более реалистичным. Поэтому задача визуализации реалистичных облаков чаще всего встречается при разработке компьютерных игр и в кинематографе~\cite{oz, hzd, frostbite}. 

Целью данной работы является реализация ПО, которое предоставляет возможность визуализировать облачное небо. При этом созданная программа должна позволить изменять параметры, влияющие на внешний вид неба.

Чтобы достигнуть поставленной цели, требуется решить следующие задачи:

\begin{itemize}
	\item описать предметную область визуализации облаков и существующие методы их представления;
	\item спроектировать ПО, позволяющее визуализировать облака;
	\item выбрать инструменты для реализации ПО;
	\item реализовать спроектированное ПО;
	\item исследовать временные характеристики реализованного ПО.
\end{itemize}