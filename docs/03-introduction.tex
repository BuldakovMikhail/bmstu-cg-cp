\chapter*{ВВЕДЕНИЕ}
\addcontentsline{toc}{chapter}{ВВЕДЕНИЕ}

Компьютерная графика представляет собой совокупность методов
и способов преобразования информации в графическое представление при
помощи ЭВМ. Без компьютерной графики не обходится ни одна современная
программа. В течении нескольких десятилетий компьютерная графика прошла
долгий путь, начиная с базовых алгоритмов, таких как вычерчивание линий
и отрезков, до построения виртуальной реальности.

Облака сильно влияют на восприятие изображения, делая его более реалистичным. Поэтому задача визуализации реалистичных облаков чаще всего встречается при разработке компьютерных игр и в кинематографе. Раньше облака в играх рисовались обычными 2D спрайтами, которые всегда повернуты в направлении камеры, но последние годы вычислительные ресурсы компьютеров позволяют рисовать облака, близкие к физически корректным, без заметных потерь в производительности.

Целью данной работы является реализация программного обеспечения, которое предоставляет возможность создавать облачное небо, состоящее из объемных, близких к физически корректным, облаков. При этом созданная программа должна позволить изменять параметры, влияющие на внешний вид неба.

Чтобы достигнуть поставленной цели, требуется решить следующие задачи:

\begin{itemize}
	\item изучить явления, происходящие в облаках;
	\item проанализировать существующие способы описания облаков;
	\item проанализировать и выбрать способы визуализации облаков;
	\item проанализировать эффекты, возникающие в облаках, и на их основе разработать модель освещения;
	\item реализовать выбранные алгоритмы;
	\item разработать программное обеспечение для отображения облачного неба.
\end{itemize}