\chapter*{ЗАКЛЮЧЕНИЕ}
\addcontentsline{toc}{chapter}{ЗАКЛЮЧЕНИЕ}

Для достижение поставленной цели были выполнены следующие задачи:
\begin{itemize}
	\item проведен анализ предметной области визуализации облаков и описаны существующие методы их представления;
	\item спроектировано ПО, позволяющее визуализировать облака;
	\item выбраны инструменты для реализации ПО;
	\item реализовано спроектированное ПО;
	\item исследованы временные характеристики ПО.
\end{itemize}

В результате исследования разработанного ПО выяснилось, что кадровая частота практически не зависит от таких параметров, как плотность облаков и уровень покрытия неба облаками и для размеров изображения $900 \times 600$ получилось достичь 32--34 кадров в секунду.
Существенное влияние на кадровую частоту оказывает размер изображения, так если для размера изображения $640 \times 360$ получилось достичь 53 кадров в секунду, то для разрешения $3840 \times 2160$ уже 3 кадра в секунду.

В результате цель работы была достигнута.


