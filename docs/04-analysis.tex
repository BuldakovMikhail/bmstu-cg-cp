\chapter{Аналитический раздел}

В данном разделе будет рассмотрена предметная область визуализации облаков.

\section{Модель облаков}
\label{math}
Облака можно представить с помощью оптической модели с рассеиванием и поглощением \cite{partmedia, frostbite, hzd, clouds}.
Оптическая модель описывает как объем излучает, отражает, рассеивает и поглощает свет \cite{rtvg}.

Для визуализации облаков, необходимо рассчитать энергию светового пучка, проходящую через облачный объем. Поскольку распространяющийся луч подвергается поглощению, необходимо рассмотреть приближение закона Бугера---Ламберта---Бера \cite{partmedia, rtvg}. Энергия светового пучка, обладающего начальной энергией $c$, и прошедшего от точки $x_0$ до $x$ внутри облака, вычисляется по формуле \eqref{eq:beers_law_app}.
\begin{equation}
	\label{eq:beers_law_app}
	c' = c \cdot e^{-\int_{x_0}^{x} k(u)\,du},
\end{equation}
где $c'$ --- итоговая яркость пучка, $k(u)$ --- коэффициент поглощения в точке пространства, а интеграл $\tau (d_1, d_2) = -\int_{d_1}^{d_2} k(u)\,du$ называют оптической глубиной \cite{partmedia, rtvg, hzd, frostbite}.

Коэффициент прозрачности в точке $u$ определяется по формуле \eqref{eq:opacity}.
\begin{equation}
	\label{eq:opacity}
	A_u = 1 - e^{-k(u)}
\end{equation}

Считается, что коэффициент поглощения $k(u)$ пропорционален оптической плотности облака в точке $u$ \cite{partmedia, rtvg, hzd, frostbite}. Тогда для того чтобы задать форму облака, необходимо задать поле плотности облака в пространстве.  


\section{Методы представления облаков}
Методы представления облаков должны определять пространственное распределение плотности облаков на сцене.

Были рассмотрены следующие методы представления облаков:
\begin{itemize}
	\item система частиц;
	\item неявное представление;
	\item объемы, ограниченные поверхностью (англ. Surface-Bounded Volumes).
\end{itemize}

\subsection{Система частиц}
\label{particles}
Система частиц --- используемый в компьютерной графике способ представления объектов, не имеющих четких геометрических границ \cite{particles1}. 
Облако можно представить системой частиц \cite{voxel}. Каждая частица в системе задается своим положением в пространстве. Пусть $x$ --- некоторая точка в пространстве, $R$ --- заранее заданный радиус восприимчивости к частицам, $f(x)$ --- функция распределения плотности от расстояния до частицы. Тогда алгоритм определения плотности в точке $x$ будет состоять из следующих шагов.

\begin{enumerate}
	\item Найти все частицы, расстояние до которых от точки $x$ не больше заданного радиуса $R$.
	\item Вычислить расстояния от заданной точки $x$ до отобранных частиц.
	\item Для каждого вычисленного расстояния найти значений функции $f(x)$.
	\item Просуммировать полученные значения функций.
\end{enumerate}

Заметим, что выбор частиц в переделах заданного радиуса $R$ требует полного перебора всех частиц. Введение иерархических структур позволяет снизить количество частиц, которые необходимо рассмотреть для отбора требуемых.


\textbf{Воксельные октодеревья}

Октодерево --- древовидная структура данных, в которой у каждого внутреннего узла ровно восемь потомков \cite{octree}.
Каждый узел октодерева задает некоторый объем пространства, а каждый потомок этого узла описывает определенную октанту данного пространства. 
В родительском узле хранятся частицы, попавшие на границы октантов, а в дочернем узле частицы, принадлежащие октанту, который сопоставляется с данным дочерним узлом. Т.~о. для отбора частиц в радиусе $R$ достаточно рассмотреть, только частицы, принадлежащие определенному узлу дерева \cite{clouds}. 

\textbf{Двоичное разбиение пространства}

Двоичное разбиение пространства является методом рекурсивного разбиения пространства на выпуклые множества гиперплоскостями \cite{neighbours}. Каждая гиперплоскость сопоставляется с узлом дерева, что позволяет построить двочиное дерево. Списки частиц хранятся в листьях такого дерева. Выполняя поиск по такому дереву можно найти частицы, которые необходимо рассмотреть для отбора.

\textbf{Иерархия ограничивающих объёмов}

В данном методе группы частиц окружаются оболочками и выстраивается иерархия в виде древовидной структуры данных (двоичное BVH-дерево). По аналогии с двоичным разбиением пространства, для отбора частиц необходимо рассмотреть только те частицы, которые заключены в оболочку, которая сопоставляется с листовым узлом BVH-дерева \cite{clouds}.


\subsection{Объемы, ограниченные поверхностью} 
Облачный объем представляется окружающей его оболочкой, обычно заданной с помощью полигональной сетки. Полигональная сетка --- это совокупность вершин, рeбер и граней, которые определяют форму многогранного объекта в пространстве \cite{mesh}.

Поскольку информация о внутренней структуре облака отсутствует, то предполагается, что объем однородный. Тогда плотность в точке $x$ определяется по следующим шагам \cite{clouds}. 
\begin{enumerate}
	\item Проверить попала ли точка в какой-нибудь из ограниченных, заданных объемов.
	\item Если точка принадлежит объему, то плотность в этой точке, равна заданной плотности объема.
	\item Если точка не принадлежит ни одному объему, то плотность в этой точке равна 0.
\end{enumerate}

\subsection{Неявное представление} 
\label{implicit}
Распространенным способом представления поля плотности облаков является использование процедурных методов \cite{frostbite, hzd}. В то время как общая форма обычно задается простыми геометрическими объектами, такими как сферы или эллипсоиды, внутренняя структура с высоким разрешением генерируется процедурно \cite{clouds}. 

Для процедурной генерации могут использоваться функции шума, такие как: шум Перлина, шум Ворлея. 

\textbf{Шум Перлина} --- это градиентный шум, состоящий из набора псевдослучайных единичных векторов (направлений градиента), расположенных в определенных точках пространства и интерполированных функцией сглаживания, значений между этими точками \cite{perlin}.

\textbf{Шум Ворлея} --- это шум полученный инверсией шума Вороного. Для генерации шума Вороного необходимо создать регулярную сетку некоторого размера и разместить в центре каждой ячейки точку. Далее используя хеш-функцию, каждая точка смещается в пределах ячейки. Теперь, чтобы рассчитать шум для любого пикселя, нужно определить в какую ячейку он попадает и какие ячейки ему смежны. Далее проходя по всем ячейкам, определяется расстояние между пикселем и смещенными точками, значение шума Вороного --- минимальное из расстояний \cite{worley}. 

В итоге, для определения плотности в некоторой точке пространства, необходимо вычислить значение функции шума в данной точке. При этом, можно вычислить значения функции в некотором объеме заранее и хранить эти вычисления в памяти. Данный метод можно использовать в комбинации с полигональной сеткой \cite{implicit, hzd, frostbite, clouds}.


\subsection*{Вывод}

В таблице \ref{tbl:repr_choice} представлено сравнение способов представления облаков. По каждому параметру составлен рейтинг: 1 --- лучший способ, 3 --- худший.

Сравнение проводится для случая, когда необходимо заполнить небо облаками. Трудоемкость вычисления плотности оценивалась без учета предварительных вычислений.

В таблице \ref{tbl:repr_choice} введены следующие обозначения:
\begin{itemize}
	\item СЧ --- система частиц с использованием октодеревьев;
	\item ООП --- объемы, ограниченные поверхностью;
	\item НП --- неявное представление, с предварительно вычисленными значениями функций шума.
\end{itemize} 

\begin{table}[h]
	\begin{center}
		\begin{threeparttable}
			\captionsetup{justification=raggedright,singlelinecheck=off}
			\caption{Сравнение методов представления облаков}
			\label{tbl:repr_choice}
			\begin{tabular}{|c|c|c|c|}
				\hline
				Характеристика &  СЧ  & ООП & НП \\
				\hline
				Реалистичность изображения &  1  & 3 & 1 \\
				\hline
				Трудоемкость вычисления плотности в точке &  3  & 2 & 1 \\
				\hline
				Требования по памяти & 2 & 1 & 3 \\
				\hline
			\end{tabular}
		\end{threeparttable}
	\end{center}
\end{table}

В результате для решения поставленной задачи был выбран метод неявного представления. Такой выбор обусловлен сравнительно низкой трудоемкостью.


\section{Анализ методов визуализации облаков}

Для визуализации облаков по их пространственному представлению, необходимо провести вычисление цвета каждого пикселя в буфере кадра. Во всех описанных алгоритмах применятся смешивание цветов. Допустим у нас есть массив из $n$ элементов, содержащий значения цвета и прозрачности в некоторой точке. Тогда смешение цвета вычисляется по формуле \eqref{eq:alpha_blending} итеративно в порядке от последнего к первому, изменяя $i$ от $n-1$ до $0$.
\begin{equation}
	\label{eq:alpha_blending}
	C'_i = C_i + (1 - A_i)C'_{i+1},
\end{equation}
где $C'_i$ --- новое значение цвета, $C_i$ и $A_i$ --- цвет и прозрачность на $i$-м шаге, $C'_{i+1}$ --- значение смешанного цвета на предыдущем шаге. Начальным значением считают $C'_n = 0$.

Формула \eqref{eq:alpha_blending} может быть изменена для прохода по массиву от первого к последнему, т.~е. $i$ от $1$ до $n$.
\begin{equation}
	\label{eq:color_blending_for}
	C'_i = C'_{i - 1} + (1 - A'_{i - 1})C_{i}
\end{equation}
\begin{equation}
	\label{eq:alpha_blending_for}
	A'_i = A'_{i - 1} + (1 - A'_{i - 1})A_{i}
\end{equation}
Начальными значениями в этом случае считают $C'_0 = 0$ и $A'_0=0$ \cite{rtvg}.


\subsection{Алгоритм Сдвиг---Деформации}
\label{slice}

Алгоритм Сдвиг---Деформации состоит из следующих шагов \cite{shearwarp}.

\begin{enumerate}
	\item Сформировать массив срезов объема плоскостями, параллельными оси координат $z$.
	\item Каждый срез из массива сдвинуть перпендикулярно оси $z$, так чтобы луч, распространяющийся в объеме был параллелен оси $z$ (данный шаг называется сдвигом). 
	\item Смешать пиксели текстур срезов в некотором временном буфере, в порядке от последнего к первому.
	\item Спроецировать временный буфер на буфер кадра (данный шаг называется деформацией).
\end{enumerate}


\subsection{Разбрызгивание текстур}
\label{splatting}
Метод разбрызгивания текстур применяется чаще всего к системам частиц, но в общем случае, под частицой, в данном контексте, подразумевается точка пространства с заданной плотностью.
Частицы могут быть визуализированы с помощью текстурированного четырехугольника, представляющего проекцию частицы на плоскость, также называемую <<пятном>> или <<отпечатком>> (от англ. splat и footprint соответственно). 

Алгоритм разбрызгивания состоит из следующих шагов, применяемых к каждому пикселю буфера кадра \cite{voxel, clouds}.
\begin{enumerate}
	\item Найти частицы, пересекаемые лучом, исходящим от наблюдателя.
	\item Смешать проекции текстур этих частиц в буфере кадра, в порядке от последней к первой.
\end{enumerate} 


\subsection{Ray Marching}
\label{ray}

Алгоритм Ray Marching состоит из следующих шагов \cite{hzd, frostbite, clouds}.
\begin{enumerate}
	\item Для каждого пикселя буфера выпускается луч от наблюдателя.
	\item Разбить луч на равные интервалы.
	\item Выполнить итеративное смешение цветов на этих интервалах.
\end{enumerate}


\subsection{Выбор метода визуализации}
В таблице \ref{tbl:rend_choice} представлено сравнение алгоритмов визуализации облаков. По каждому параметру составлен рейтинг: 1 --- лучший способ, 3 --- худший.

Сравнение проводится для случая, когда необходимо заполнить небо облаками, перспектива отсутствует. 

В таблице \ref{tbl:rend_choice} введены следующие обозначения:
\begin{itemize}
	\item СД --- алгоритм сдвиг---деформации;
	\item РТ --- алгоритм разбрызгивания текстур;
	\item RM --- алгоритм Ray Marching.
\end{itemize} 

\begin{table}[h]
	\begin{center}
		\begin{threeparttable}
			\captionsetup{justification=raggedright,singlelinecheck=off}
			\caption{Сравнение алгоритмов визуализации облаков}
			\label{tbl:rend_choice}
			\begin{tabular}{|c|c|c|c|}
				\hline
				Характеристика &  СД  & РТ & RM \\
				\hline
				Визуализация облаков сложных форм &  3  & 2 & 1 \\
				\hline
				Трудоемкость построения изображения &  1  & 3 & 2 \\
				\hline
				Требования по памяти & 3 & 1 & 1 \\
				\hline
			\end{tabular}
		\end{threeparttable}
	\end{center}
\end{table}

В результате был выбран алгоритм Ray Marching, т.~к. данный алгоритм подходит для визуализации облаков любой формы.

\section{Модель освещения}

Солнце считается точечным источником света. Угол падения света считается одинаковым для всех точек пространства в пределах сцены.

Световая энергия луча, распространяющегося в облачном объеме, может быть поглощена или отражена некоторой частицей. При этом облака являются анизатропной средой, т.~е. отраженные фотоны распространяются в разных направлениях неравномерно \cite{clouds, partmedia, hzd, frostbite}.

Поглощение луча описывается законом Бугера---Ламберта---Бера, и рассчитывается по формуле \eqref{eq:bl}.
\begin{equation}
	\label{eq:bl}
	BL = e ^ {-d}
\end{equation}
где $ d $ -- накопленная плотность.


Для аппроксимации неравномерности распространения отраженного света используют фазовую функцию \cite{partmedia, hzd, frostbite, clouds}.
Фазовая функция описывает вероятность отражения фотона от рассеивающего объекта под определенным углом. 
В качестве фазовой функции для облаков используется функция Хеньи --- Гринштейна \eqref{eq:hg}.  
\begin{equation}
	\label{eq:hg}
	p(\theta) = \frac{1}{4\pi} \frac{1 - g^2}{(1 + g^2 - 2g\cos \theta)^{3/2}},
\end{equation}
где $ g $ --- варьируемый параметр, причем $ -1 \leq g \leq 1 $, а $\theta$ --- угол между падающим и отраженным лучами \cite{clouds}.


Из распределения отраженного света фазовой функцией видно, что луч, после отражения будет распространяться вероятнее всего в направлении близком к исходному. 
Таким образом, чтобы направление луча изменилось на 180 градусов необходим достаточно длинный путь внутри облака, что приводит к большой накопленной плотности, в результате чего по формуле \ref{eq:bl} энергия луча будет близка к нулю. 
В результате чего, наблюдатель, выпускающий луч в направлении близком к направлению падения света, будет видеть нереалистично темные облака \cite{hzd}. Для того чтобы этого не происходило вводится поправка по формуле \eqref{eq:sugar_powder}.

\begin{equation}
	\label{eq:sugar_powder}
	PSE = 1 - e ^ {-2 d}
\end{equation}

Для расчета энергии луча, испущенного солнцем, в некоторой точке пространства, необходимо учесть путь этого луча внутри облака до рассматриваемой точки.
Для этого плотность $d$, вычисляется на нескольких точках, выбранных на луче.
В итоге, энергия света, в некоторой точке пространства вычисляется по формуле \eqref{eq:light_model}.
\begin{equation}
	\label{eq:light_model}
	E = 2 \cdot e ^ {-d} \cdot (1 - 2 e ^ {-2d}) \cdot \frac{1}{4\pi} \frac{1 - g^2}{(1 + g^2 - 2g\cos \theta)^{3/2}}
\end{equation}
где $ g $ -- некоторая константа в функции Хеньи-Гринштейна, а $ \theta $ -- угол между лучами взгляда и падения света.


\section*{Вывод}

В данном разделе был проведен анализ способов хранения поля плотности облаков, методов визуализации и модели освещения, которые возможно использовать для решения поставленных задач. В итоге была выбрана связка из неявного представления облаков с алгоритмом Ray Marching, так как такой подход позволяет достичь высокой реалистичности, а также точности построенного изображения. 








