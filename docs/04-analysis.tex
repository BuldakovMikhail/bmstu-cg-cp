\chapter{Аналитический раздел}

В данном разделе будет рассмотрена предметная область визуализации облаков.

\section{Модель облаков}
\label{math}
Облака можно представить с помощью оптической модели с рассеиванием и затенением \cite{partmedia, frostbite, hzd, clouds}.
Оптическая модель описывает как объем излучает, отражает, рассеивает и поглощает свет \cite{gatech}. Модель включающая рассеивание и затенение учитывает только рассеивание света частицами облака и поглощение этими частицами света.

Для визуализации облаков, необходимо рассчитать энергию светового пучка, проходящую через облачный объем. Поскольку распространяющийся луч подвергает поглощению, необходимо рассмотреть приближение закона Бугера---Ламберта---Бера \cite{partmedia, gatech}. Энергия светового пучка, обладающего начальной энергией $c$, и прошедшего от точки $x_0$ до $x$ внутри облака, вычисляется по формуле \eqref{eq:beers_law_app}.
\begin{equation}
	\label{eq:beers_law_app}
	c' = c \cdot e^{-\int_{x_0}^{x} k(u)\,du},
\end{equation}
где $c'$ --- итоговая яркость пучка, $k(u)$ --- коэффициент поглощения в точке пространства, а интеграл $\tau (d_1, d_2) = -\int_{d_1}^{d_2} k(u)\,du$ называют оптической глубиной \cite{partmedia, gatech, hzd, frostbite}.

Коэффициент прозрачности в точке $u$ определяется по формуле \eqref{eq:opacity}.
\begin{equation}
	\label{eq:opacity}
	A_u = 1 - e^{-k(u)}
\end{equation}

Считается, что коэффициент поглощения $k(u)$ пропорционален плотности облака в точке $u$ \cite{partmedia, gatech, hzd, frostbite}. Тогда для того чтобы задать форму облака, необходимо задать поле плотности облака в пространстве.  


\section{Методы представления облаков}
Методы представления облаков должны определять пространственное распределение плотности облаков на сцене.

Были рассмотрены следующие методы представления облаков:
\begin{itemize}
	\item система частиц;
	\item неявное представление;
	\item объемы, ограниченные поверхностью (англ. Surface-Bounded Volumes).
\end{itemize}

\subsection{Система частиц}
\label{particles}
Система частиц --- используемый в компьютерной графике способ представления объектов, не имеющих четких геометрических границ \cite{particles1}. 
Облако можно представить системой частиц \cite{voxel}. Каждая частица в системе задается своим положением в пространстве. Пусть $x$ --- некоторая точка в пространстве, $R$ --- заранее заданный радиус восприимчивости к частицам, $f(x)$ --- функция распределения плотности от расстояния до частицы. Тогда алгоритм определения плотности в точке $x$ будет состоять из следующих шагов.

\begin{enumerate}
	\item Найти все частицы, расстояние до которых от точки $x$ не больше заданного радиуса $R$.
	\item Вычислить расстояния от заданной точки $x$ до отобранных частиц.
	\item Для каждого вычисленного расстояния найти значений функции $f(x)$.
	\item Просуммировать полученные значения функций.
\end{enumerate}

Заметим, что выбор частиц в переделах заданного радиуса $R$ требует полного перебора всех частиц. Введение иерархических структур позволяет снизить количество частиц, которые необходимо рассмотреть для отбора требуемых.


\textbf{Воксельные октодеревья}

Октодерево --- древовидная структура данных, в которой у каждого внутреннего узла ровно восемь потомков \cite{octree}.
Каждый узел октодерева задает некоторый объем пространства, а каждый потомок этого узла описывает определенную октанту данного пространства. 
В родительском узле хранятся частицы, попавшие на границы октантов, а в дочернем узле частицы, принадлежащие октанту, который сопоставляется с данным дочерним узлом. Т.~о. для отбора частиц в радиусе $R$ достаточно рассмотреть, только частицы, принадлежащие определенному узлу дерева \cite{clouds}. 

\textbf{Двоичное разбиение пространства}

Двоичное разбиение пространства является методом рекурсивного разбиения пространства на выпуклые множества гиперплоскостями \cite{neighbours}. Каждая гиперплоскость сопоставляется с узлом дерева, что позволяет построить двочиное дерево. Списки частиц хранятся в листьях такого дерева. Выполняя поиск по такому дереву можно найти частицы, которые необходимо рассмотреть для отбора.

\textbf{Иерархия ограничивающих объёмов}

В данном методе группы частиц окружаются оболочками и выстраивается иерархия в виде древовидной структуры данных (двоичное BVH-дерево). По аналогии с двоичным разбиением пространства, для отбора частиц необходимо рассмотреть только те частицы, которые заключены в оболочку, которая сопоставляется с листовым узлом BVH-дерева \cite{clouds}.


\subsection{Объемы, ограниченные поверхностью} 
Облачный объем представляется окружающей его оболочкой, обычно заданной с помощью полигональной сетки. Полигональная сетка --- это совокупность вершин, рeбер и граней, которые определяют форму многогранного объекта в пространстве \cite{mesh}.

Поскольку информация о внутренней структуре облака отсутствует, то предполагается, что объем однородный. Тогда плотность в точке $x$ определяется по следующим шагам \cite{clouds}. 
\begin{enumerate}
	\item Проверить попала ли точка в какой-нибудь из ограниченных, заданных объемов.
	\item Если точка принадлежит объему, то плотность в этой точке, равна заданной плотности объема.
	\item Если точка не принадлежит ни одному объему, то плотность в этой точке равна 0.
\end{enumerate}

\subsection{Неявное представление} 
\label{implicit}
Распространенным способом представления поля плотности облаков является использование процедурных методов \cite{frostbite, hzd}. В то время как общая форма обычно задается простыми геометрическими объектами, такими как сферы или эллипсоиды, внутренняя структура с высоким разрешением генерируется процедурно \cite{clouds}. 

Для процедурной генерации могут использоваться функции шума, такие как: шум Перлина, шум Ворлея. 

\textbf{Шум Перлина} --- это градиентный шум, состоящий из набора псевдослучайных единичных векторов (направлений градиента), расположенных в определенных точках пространства и интерполированных функцией сглаживания, значений между этими точками \cite{perlin}.

\textbf{Шум Ворлея} --- это шум полученный инверсией шума Вороного. Для генерации шума Вороного необходимо создать регулярную сетку некоторого размера и разместить в центре каждой ячейки точку. Далее используя хеш-функцию, каждая точка смещается в пределах ячейки. Теперь, чтобы рассчитать шум для любого пикселя, нужно определить в какую ячейку он попадает и какие ячейки ему смежны. Далее проходя по всем ячейкам, определяется расстояние между пикселем и смещенными точками, значение шума Вороного --- минимальное из расстояний \cite{worley}. 

В итоге, для определения плотности в некоторой точке пространства, необходимо вычислить значение функции шума в данной точке. При этом, можно вычислить значения функции в некотором объеме заранее и хранить эти вычисления в памяти. Данный метод можно использовать в комбинации с полигональной сеткой \cite{implicit, hzd, frostbite, clouds}.


\subsection*{Вывод}

В таблице \ref{tbl:repr_choice} представлено сравнение способов представления облаков. По каждому параметру составлен рейтинг: 1 --- лучший способ, 3 --- худший.

Сравнение проводится для случая, когда необходимо заполнить небо облаками. Трудоемкость вычисления плотности оценивалась без учета предварительных вычислений.

В таблице \ref{tbl:repr_choice} введены следующие обозначения:
\begin{itemize}
	\item СЧ --- система частиц с использованием октодеревьев;
	\item ООП --- объемы, ограниченные поверхностью;
	\item НП --- неявное представление, с предварительно вычисленными значениями функций шума.
\end{itemize} 

\begin{table}[h]
	\begin{center}
		\begin{threeparttable}
			\captionsetup{justification=raggedright,singlelinecheck=off}
			\caption{Сравнение методов представления облаков}
			\label{tbl:repr_choice}
			\begin{tabular}{|c|c|c|c|}
				\hline
				Характеристика &  СЧ  & ООП & НП \\
				\hline
				Реалистичность изображения &  1  & 3 & 1 \\
				\hline
				Трудоемкость вычисления плотности в точке &  3  & 2 & 1 \\
				\hline
				Требования по памяти & 2 & 1 & 3 \\
				\hline
			\end{tabular}
		\end{threeparttable}
	\end{center}
\end{table}

В результате для решения поставленной задачи был выбран метод неявного представления. Такой выбор обусловлен сравнительно низкой трудоемкостью.


\section{Анализ методов визуализации облаков}

Для визуализации облаков по их пространственному представлению, необходимо провести вычисление цвета каждого пикселя в буфере кадра. Во всех описанных алгоритмах применятся смешивание цветов. Допустим у нас есть массив из $n$ элементов, содержащий значения цвета и прозрачности в некоторой точке. Тогда смешение цвета вычисляется по формуле \eqref{eq:alpha_blending} итеративно в порядке от последнего к первому, изменяя $i$ от $n-1$ до $0$.
\begin{equation}
	\label{eq:alpha_blending}
	C'_i = C_i + (1 - A_i)C'_{i+1},
\end{equation}
где $C'_i$ --- новое значение цвета, $C_i$ и $A_i$ --- цвет и прозрачность на $i$-м шаге, $C'_{i+1}$ --- значение смешанного цвета на предыдущем шаге. Начальным значением считают $C'_n = 0$.

Формула \eqref{eq:alpha_blending} может быть изменена для прохода по массиву от первого к последнему, т.~е. $i$ от $1$ до $n$.
\begin{equation}
	\label{eq:color_blending_for}
	C'_i = C'_{i - 1} + (1 - A'_{i - 1})C_{i}
\end{equation}
\begin{equation}
	\label{eq:alpha_blending_for}
	A'_i = A'_{i - 1} + (1 - A'_{i - 1})A_{i}
\end{equation}
Начальными значениями в этом случае считают $C'_0 = 0$ и $A'_0=0$ \cite{gatech}.


\subsection{Алгоритм Сдвиг---Деформации}
\label{slice}

Алгоритм Сдвиг---Деформации состоит из следующих шагов \cite{shearwarp}.

\begin{enumerate}
	\item Сформировать массив срезов объема плоскостями, параллельными оси координат $z$.
	\item Каждый срез из массива сдвинуть перпендикулярно оси $z$, так чтобы луч, распространяющийся в объеме был параллелен оси $z$ (данный шаг называется сдвигом). 
	\item Смешать пиксели текстур срезов в некотором временном буфере, в порядке от последнего к первому.
	\item Спроецировать временный буфер на буфер кадра (данный шаг называется деформацией).
\end{enumerate}


\subsection{Разбрызгивание текстур}
\label{splatting}
Метод разбрызгивания текстур применяется чаще всего к системам частиц, но в общем случае, под частицой, в данном контексте, подразумевается точка пространства с заданной плотностью.
Частицы могут быть визуализированы с помощью текстурированного четырехугольника, представляющего проекцию частицы на плоскость, также называемую <<пятном>> или <<отпечатком>> (от англ. splat и footprint соответственно). 

Алгоритм разбрызгивания состоит из следующих шагов, применяемых к каждому пикселю буфера кадра \cite{voxel, clouds}.
\begin{enumerate}
	\item Найти частицы, пересекаемые лучом, исходящим от наблюдателя.
	\item Смешать проекции текстур этих частиц в буфере кадра, в порядке от последней к первой.
\end{enumerate} 


\subsection{Ray Marching}
\label{ray}

Алгоритм Ray Marching состоит из следующих шагов \cite{hzd, frostbite, clouds}.
\begin{enumerate}
	\item Для каждого пикселя буфера выпускается луч от наблюдателя.
	\item Разбить луч на равные интервалы.
	\item Выполнить итеративное смешение цветов на этих интервалах.
\end{enumerate}


\subsection{Выбор метода визуализации}
В таблице \ref{tbl:rend_choice} представлено сравнение алгоритмов визуализации облаков. По каждому параметру составлен рейтинг: 1 --- лучший способ, 3 --- худший.

Сравнение проводится для случая, когда необходимо заполнить небо облаками, перспектива отсутствует. 

В таблице \ref{tbl:rend_choice} введены следующие обозначения:
\begin{itemize}
	\item СД --- алгоритм сдвиг---деформации;
	\item РТ --- алгоритм разбрызгивания текстур;
	\item RM --- алгоритм Ray Marching.
\end{itemize} 

\begin{table}[h]
	\begin{center}
		\begin{threeparttable}
			\captionsetup{justification=raggedright,singlelinecheck=off}
			\caption{Сравнение алгоритмов визуализации облаков}
			\label{tbl:rend_choice}
			\begin{tabular}{|c|c|c|c|}
				\hline
				Характеристика &  СД  & РТ & RM \\
				\hline
				Визуализация облаков сложных форм &  3  & 2 & 1 \\
				\hline
				Трудоемкость построения изображения &  1  & 3 & 2 \\
				\hline
				Требования по памяти & 3 & 1 & 1 \\
				\hline
			\end{tabular}
		\end{threeparttable}
	\end{center}
\end{table}

В результате был выбран алгоритм Ray Marching, т.~к. данный алгоритм подходит для визуализации облаков любой формы.

\section{Модель освещения}

Для реалистичной визуализации облаков, модель освещения должна аппроксимировать следующие эффекты: множественное рассеяние и направленное освещение в облаках, <<серебряное обрамление>> (эффект возникающий, когда мы смотрим на солнце через облака), и темные границы облака, когда мы смотрим на облако в направлении от солнца. 

C фотоном попавшим в облако, может произойти три события:
\begin{enumerate}
	\item он может поглотиться частицей воды или другой частицей, не относящейся к облаку, например пылью;
	\item в ходе отражений, он выходит из облака и движется в сторону наблюдателя. Назовем это внутренним рассеянием;
	\item в ходе отражений, он выходит из облака и движется в сторону противоположную от наблюдателя. Назовем это внешним рассеянием.
\end{enumerate}

Для моделирования явления <<серебряного обрамления>>, используются фазовые функции, одной из которых, является функция Хеньи-Гринштейна \eqref{HG}.

Закон Бера -- это модель поглощения, а это означает, что он описывает то, как световая энергия поглощается при распространении вглубь облака. Но такая модель не учитывает важный световой эффект, который связан с внутренним рассеянием на обращенных к солнцу сторонах облаков. Этот эффект проявляется в виде темных краев на облаках, когда направление луча обзора приближается к направлению луча света. Этот эффект наиболее заметен в круглых, плотных областях облаков, настолько, что складки между каждой выпуклостью кажутся ярче, чем сама выпуклость, которая находится ближе к солнцу. Эти результаты кажутся полной противоположностью тому, что моделирует закон Бера.  Напомним, что большее количество света рассеивается вперед, вдоль исходного направления светового луча из-за прямого рассеяния. Должна существовать относительно большая оптическая толщина, чтобы фотон мог повернуться на 180 градусов. Пути около границы облака не проходят через достаточно большую оптическую толщину, чтобы полностью обернуть заметную часть фотонов. Пути, которые имеют оптическую толщину, достаточную для поворота фотона на 180 градусов, почти всегда находятся внутри облака, поэтому закон Бера заглушит этот вклад до того, как частица света покинет облако. Щели и трещины являются исключением, они создают <<окно>>, через которое фотоны могут покинуть облако по пути с малой плотностью, делая расщелины ярче, чем окружающие их выпуклости. Для учета этого эффекта в закон Бера вносят поправку:

\begin{equation}
	\label{sugar_powder}
	PSE = 1 - e ^ {-2 d}
\end{equation}

Приближено, считаем световую энергию, накопленную лучом, таким образом:
\begin{equation}
	\label{bsp}
	LE = 2 * BL * PSE
\end{equation}
где $ LE $ - накопленная энергия света. 

В итоге, полностью наша модель освещения описывается следующим образом:
\begin{equation}
	\label{light_model}
	E = 2 * e ^ {-d} * (1 - 2 e ^ {-2d}) * \frac{1}{4\pi} \frac{1 - g^2}{(1 + g^2 - 2g\cos \theta)^{3/2}}
\end{equation}
где $ g $ -- некоторая константа в функции Хеньи-Гринштейна, а $ \theta $ -- угол между лучами взгляда и света.

\subsection{Расчет освещения}
Очевидный способ найти количество падающего света -- измерить переносимую световую энергию внутри облака между рассматриваемой точкой и солнцем. Однако на освещенность любой точки облака сильно влияет освещенность в областях вокруг нее в направлении источника света, т.~е. нам необходимо рассмотреть конус с вершиной в рассматриваемой точке. Внутри конуса мы выбираем случайным образом, некоторое количество точек, в которых рассчитываем плотность. Суммируя плотности в точках, мы получаем плотность накопленную в конусе, далее используем ее, чтобы вычислить световую энергию в рассматриваемой точке.

Для более грубого расчета освещения, можно накапливать плотность не внутри конуса, а в точках на луче, который начинается в освещаемой точке и направлен к источнику света. 


%%% --------------------------------- COMMENTED -----------------------------
\mycomment{
	\section{Техники освещения}
	
	\subsection{Аппроксимация однократного рассеивания}
	\begin{itemize}
		\item На основе срезов: нарезанный объем используется для создания объемной карты освещения, хранящей яркость света. Этот так называемый <<световой объем>> ориентирован так, что свет проходит перпендикулярно или параллельно срезам из которых он состоит, что обеспечивает прямое распространение света от среза к срезу \cite{schpok};
		\item На основе частиц. Используется отбрасывания теней в качестве аппроксимации однократного рассеяния для освещения системы частиц. Сцена преобразуется таким образом, чтобы точка наблюдения совпала с источником света. Кадровый буфер графического процессора используется в качестве карты теней. Частицы сортируются и обрабатываются в направлении спереди назад относительно источника света. Для каждой частицы значение тени рассчитывается с помощью коэффициента затухания, который в свою очередь рассчитывается итеративно, т.~е. коэффициент затухания в некоторой частице равен произведению коэффициентов всех частиц, через который прошел свет на пути к данной частице \cite{voxel}.
	\end{itemize}
	
	\subsection{Диффузия света}
	
	В оптически плотных средах, когда свет сталкивается с множеством частиц, диффузия является допустимым приближением для описания распределения света внутри таких сред. Однако это приближение становится менее точным в неоднородных средах или на их границах. 
	Возможно комбинировать диффузию с анизотропным рассеиванием, чтобы учитывать неоднородности в среде, например, чтобы воспроизвести такой эффект, как <<серебряное обрамление>> (англ. silver-lining) в облаках. <<Серебряное обрамление>> - это светящаяся граница облаков, которая может наблюдаться при правильных условиях освещения.
	Таким образом, данная модель может быть эффективной для создания более реалистичных визуальных эффектов в облаках, но требует значительных вычислительных ресурсов и рендеринг может занимать несколько часов \cite{clouds}.
	
	
	\subsection{Трассировка пути}
	
	Подход заключается в том, что для каждого пикселя на изображении запускаются сотни или даже тысячи лучей в сцену, которые отслеживаются до тех пор, пока они не достигнут источника света или не будут отражены обратно в камеру. Метод <<Трассировки пути>> обеспечивает физически корректные результаты, но, как правило, страдает от шума и низкой скорости работы.
	
	Различные структуры данных используются для разделения объемных данных на подмножества для эффективного обхода и отслеживания лучей. В качестве таких структур данных могут выступать kd-деревья или регулярные сетки. Такие ускоренные методы позволяют сократить время рендеринга и снизить уровень шума, делая метод <<Трассировки пути>> более применимым для сложных объемных данных, таких как облака, и позволяют получить более качественные и реалистичные результаты \cite{clouds}.
	
	\subsection{Фотонные карты}
	
	Фотоны в данном методе — это частицы, переносящие некоторую дискретную порцию световой энергии. На начальном этапе фотоны испускаются из источника света в соответствии с распределением световой энергии у данного источника. Например известно, что точечный или сферический источник света испускают свет изотропно во всех направлениях. В  процессе  трассировки  фотоны  ударяются  о  различные  поверхности  сцены.  В 
	зависимости от свойств материала поверхности, с ними могут происходить разные события: 
	прохождение через поверхность, полное поглощение, зеркальное отражение либо диффузное отражение. При  взаимодействии  фотона  с  диффузным  объектом  запись  о  фотоне 
	сохраняется в обычном списке или массиве. 
	
	После  завершения  трассировки  фотонов  необходимо  произвести  непосредственно 
	построение  фотонной  карты,  которая  представляет  собой  специальную  структуру 
	распределения  фотонов  на  поверхностях  сцены.  Данный  этап  позволяет  избежать 
	значительных вычислительных затрат, поскольку если бы фотоны просто сохранялись в виде 
	массива  или  списка,  то  k  ближайших  фотонов  к  текущему  каждый  раз  приходилось  бы 
	вычислять перебором, что значительно снижает быстродействие алгоритма. Для хранения фотонов используется kd-дерево.
	
	Далее применяется обычная трассировка лучей из камеры, но в местах пересечения лучей с объектами находятся k ближайших фотонов, их энергия суммируется и делится на площадь сферы с радиусом равным расстоянию да самого дальнего из этих фотонов. Полученная величина добавляется к общей энергии собранной лучом. Важным  достоинством  метода  фотонных  карт  является  тот  факт,  что  расчеты  не 
	зависят  от  положения  камеры,  что  позволяет во  многих  случаях  рассчитать  освещенность 
	всего  один  раз \cite{photon}.
	
	\subsection{Глобальная модель освещения}
	
	Для реализации явлений межоблачного затенения и непрямого освещения используется глобальная модель освещения.
	
	Представление объектов сцены с помощью сфер, позволяет быстро вычислять тени, например, используя сферические функции, или путем накопления теней в пространстве изображения \cite{clouds}.
	
	Непрямое освещение: освещенность точки на поверхности оценивается с помощью освещенности соседних точек в нескольких направлениях, ограниченных конусами.
}




%%% --------------------------------- COMMENTED -----------------------------
\mycomment{
	\section{Существующие примеры моделирования объемных облаков}
	
	Использование процедурной генерации описанной в пункте \ref{implicit}, считается современным стандартом моделирования облаков в компьютерных играх и фильмах. Этот метод применялся в:
	\begin{itemize}
		\item Компьютерной игре <<Horizon Zero Dawn>> \cite{hzd};
		\item Игровом движке Frostbite \cite{frostbite};
	\end{itemize}
	
	\img{75mm}{hzdimg}{Облака созданные с помощью процедурной генерации и погодных карт в игре <<Horizon Zero Dawn>>}
}


\section*{Вывод}

В данном разделе был проведен анализ способов хранения поля плотности облаков, методов визуализации и модели освещения, которые возможно использовать для решения поставленных задач. В итоге была выбрана связка из неявного представления облаков с алгоритмом Ray Marching, так как такой подход позволяет достичь высокой реалистичности, а также точности построенного изображения. 








