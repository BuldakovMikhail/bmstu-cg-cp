\chapter{Исследовательский раздел}

В данном разделе будет проведено исследование кадровой частоты от различных параметров.

\section{Технические характеристики}

Технические характеристики устройства, на котором выполнялись замеры по времени, следующие:
\begin{itemize}
	\item процессор: AMD Ryzen 5 4600H 3 ГГц~\cite{amd};
	\item видеокарта: NVIDIA GeForce GTX 1650 Ti~\cite{nvidia};
	\item оперативная память: 16 ГБайт;
	\item операционная система: Windows 10 Pro 64-разрядная система версии 22H2~\cite{windows}.
\end{itemize}

При замерах времени ноутбук был включен в сеть электропитания и был нагружен только системными приложениями.

\section{Время выполнения реализаций алгоритмов}

Целью исследования является определение зависимости кадровой частоты от размера изображения, плотности облаков и уровня покрытия неба облаками.

Замер времени между кадрами проводился с помощью \textit{ModernGL}~\cite{moderngl}. Для расчета количества кадров в секунду, т.~е. кадровой частоты, производился подсчет кадров и времени между ними на протяжении 100 секунд, затем, подсчитанное количество кадров делилось на суммарное время.

В результате зависимость количества кадров в секунду от плотности облаков представлена в таблице~\ref{tbl:densfps}. Для данного исследования размер изображения $900 \times 600$, уровень покрытия $1.2$.

\begin{table}[H]
	\begin{center}
		\begin{threeparttable}
			\captionsetup{justification=raggedright,singlelinecheck=off}
			\caption{Зависимость количества кадров в секунду от плотности}
			\label{tbl:densfps}
			\begin{tabular}{|c|c|}
				\hline
				Плотность &  Количество кадров в секунду \\
				\hline
					0 & 33 \\
					\hline
					20 & 33 \\
					\hline
					40 & 33 \\
					\hline
					60 & 32 \\
					\hline
					80 & 33 \\
					\hline
					100 & 33 \\
					\hline
					120 & 33 \\
					\hline
					140 & 33 \\
				\hline
			\end{tabular}
		\end{threeparttable}
	\end{center}
\end{table}


Зависимость количества кадров в секунду от покрытия неба облаками представлена в таблице~\ref{tbl:covfps}. Для данного исследования размер изображения $900 \times 600$, плотность облаков 150.

\begin{table}[H]
	\begin{center}
		\begin{threeparttable}
			\captionsetup{justification=raggedright,singlelinecheck=off}
			\caption{Зависимость количества кадров в секунду от уровня покрытия неба}
			\label{tbl:covfps}
			\begin{tabular}{|c|c|}
				\hline
				Уровень покрытия &  Количество кадров в секунду \\
					\hline
				1.0 & 34 \\
				\hline
				1.2 & 33 \\
				\hline
				1.4 & 32 \\
				\hline
				1.6 & 32 \\
				\hline
				1.8 & 32 \\
				\hline
				2.0 & 32 \\
				\hline
			\end{tabular}
		\end{threeparttable}
	\end{center}
\end{table}


Зависимость количества кадров в секунду от разрешения экрана представлена в таблице~\ref{tbl:resfps}. Для данного исследования плотность облаков 150, уровень покрытия $1.2$.

\begin{table}[H]
	\begin{center}
		\begin{threeparttable}
			\captionsetup{justification=raggedright,singlelinecheck=off}
			\caption{Зависимость количества кадров в секунду от размера изображения}
			\label{tbl:resfps}
			\begin{tabular}{|c|c|c|c|}
				\hline
				№ & Ширина & Высота & Количество кадров в секунду \\
				\hline
				1 & 640 & 360 & 53 \\
				\hline
				2 & 800 & 480 & 37 \\
				\hline
				3 & 1280 & 720 & 16 \\
				\hline
				4 & 1920 & 1080 & 8 \\
				\hline
				5 & 3840 & 2160 & 3 \\
				\hline
			\end{tabular}
		\end{threeparttable}
	\end{center}
\end{table}

На рисунке~\ref{img:fpsres} изображен график зависимости количества кадров в секунду от размера изображения.

\includeimage
{fpsres} % Имя файла без расширения (файл должен быть расположен в директории inc/img/)
{f} % Обтекание (без обтекания)
{H} % Положение рисунка (см. figure из пакета float)
{1\textwidth} % Ширина рисунка
{График зависимости количества кадров в секунду от размера изображения} % Подпись рисунка

\section*{Вывод}

В результате исследования программного обеспечения выяснилось, что в среднем для размера изображения $900 \times 600$ получается генерировать 32--34 кадра в секунду.

По рисунку~\ref{img:fpsres} и таблицам~\ref{tbl:densfps},~\ref{tbl:covfps} можно сделать вывод, что на кадровую частоту существенно влияет только размер изображения, что соответствует оценке трудоемкости $O(NM)$ алгоритма Ray Marching. 
С ростом размеров изображения кадровая частота уменьшается. Такой результат обусловлен тем, что в реализации алгоритма Ray Marching луч выпускается для каждого пикселя, т.~е. чем больше изображение, тем больше лучей необходимо испустить. При распространении луча в атмосфере выполняется фиксированное количество шагов, на каждом из которых, необходимо вычислять плотность. Плотность же, за счет сохранения значений функций шума, вычисляется за константное время, которое не зависит ни от каких параметров.