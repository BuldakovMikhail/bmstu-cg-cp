\chapter{Технологическая часть}

В данном разделе будут представлены средства разработки программного обеспечения, детали реализации и процесс сборки разрабатываемого
программного обеспечения.

\section{Выбор средств реализации}

В качестве языка программирования для разработки программного обеспечения был выбран язык Python версии 3 \cite{python}. Данный выбор обусловлен тем, что данный язык предоставляет весь функционал требуемый для решения поставленной задачи. 

Визуализации производится с помощью библиотеки ModernGL \cite{modeengl}. Данная библиотека дает возможность производить вычисления параллельно на видеокарте, что позволяет ускорить работу программы в разы. 


\section{Реализация алгоритмов}

На листингах \ref{lst:density} -- \ref{lst:lightmodel}, приведенных ниже, описаны основные алгоритмы используемые для визуализации облаков на языке шейдеров GLSL.

\clearpage

\begin{lstinputlisting}[
	caption={Реализация функции определяющей плотность в точке},
	language=GLSL,
	label={lst:density},
	linerange={1-172,311-311}
	]{./assets/code/density.glsl}
\end{lstinputlisting}


\clearpage

\begin{lstinputlisting}[
	caption={Реализация алгоритма Ray Marching},
	language=GLSL,
	label={lst:raymarching},
	linerange={1-172,311-311}
	]{./assets/code/raymarching.glsl}
\end{lstinputlisting}


\clearpage

\begin{lstinputlisting}[
	caption={Реализация модели освещения},
	language=GLSL,
	label={lst:lightmodel},
	linerange={1-172,311-311}
	]{./assets/code/lightenergy.glsl}
\end{lstinputlisting}


\section{Вывод}

В данном разделе были представлены средства разработки программного обеспечения и детали реализации. В итоге был получен рендер следующего кадра:

\img{75mm}{screen}{Кадр, облачного неба, созданный с помощью описанных алгоритмов}
