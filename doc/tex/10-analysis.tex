\chapter{Аналитическая часть}

\section{Методы представления облаков}

Для представления объемных облаков применяются следующие методы:
\begin{itemize}
	\item Система частиц;
	\item Набор вокселей;
	\item Участвующая среда (Participating media);
\end{itemize}

\subsection{Система частиц}

Система частиц - используемый в компьютерной графике способ представления объектов, не имеющих четких геометрических границ.
На экране монитора частица может выглядеть по-разному, однако нет смысла
в моделировании одной частицы. На практике моделируют большое
количество частиц, которые взаимодействуют между собой и
окружением. Совокупность частиц и применяемых к ним законов становится отдельным объектом,
который называется система частиц \cite{particles1}. Т.~к. с физической точки зрения облако и есть взвешенные в атмосфере частицы воды и/или льда, то 
облака, которые моделируются системой частиц ведут себя достаточно реалистично. В зависимости от задачи, можно вводить различные правила взаимодействия частиц.
Такой подход позволяет моделировать затухание и рождение облака, при этом разработка программного обеспечения достаточно проста. Но для моделирования одного облака необходимо большое количество частиц, что означает высокие вычислительные затраты и как следствие, визуализиация облачного неба в режиме реального времени, является невыполнимой задачей. К тому же облака, созданные с помощью системы частиц выглядят не слишком правдоподобно.  

\subsection{Набор вокселей} 

Воксель - элемент объемного изображения, содержащий значение элемента растра в трехмерном пространстве.
Представление облака с помощью вокселей позволяет использовать клеточные автоматы для создания анимаций. При этом необходимо все пространство моделирования разделить на воксели. Воксели
соответствуют ячейкам, используемым в клеточном автомате. В каждой ячейке
назначаются три логические переменные: пар/влажность (hum), облачность (cld)
и событие фазового перехода (или активации) (act). Состояние каждой
переменной равно либо 0, либо 1. Эволюция облака моделируется путем
применения простых правил перехода на каждом временном шаге. Правила
перехода представляют собой формирование, исчезновение и адвекцию облака ветрами.
Такой подход позволяет создавать реалистично выглядящие и ведущие себя облака. Но по-прежнему, создание облачного неба, требует большое количество ресурсов, для хранения и вычисления состояний вокселей, что делает невозможным моделирование неба в режиме реального времени \cite{voxel}.

\subsection{Участвующая среда} 

Участвующая среда — это термин, используемый для описания объемов, заполненных частицами. В качестве этих частиц могут быть: капли воды, кристаллы льда, пыль, молекулы. В зависимости от своего состава среда будет по-разному взаимодействовать со светом, проходящим через нее и отражающимся от частиц, что обычно называют светорассеянием. Плотность частиц на единицу объема также может изменяться в пространстве. Например, капли воды равномерно распределенные по большому объему, с малой плотностью, могут представлять туман, тогда как локальное, плотное распределение частиц в объеме, может представлять облака.



