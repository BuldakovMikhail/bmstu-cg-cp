\chapter{Аналитическая часть}

\section{Методы представления облаков}

Для представления объемных облаков применяются следующие методы:
\begin{itemize}
	\item Система частиц;
	\item Набор вокселей;
	\item Участвующая среда (англ. Participating media);
\end{itemize}

\subsection{Система частиц}

Система частиц - используемый в компьютерной графике способ представления объектов, не имеющих четких геометрических границ.
На экране монитора частица может выглядеть по-разному, однако нет смысла
в моделировании одной частицы. На практике моделируют большое
количество частиц, которые взаимодействуют между собой и
окружением. Совокупность частиц и применяемых к ним законов становится отдельным объектом,
который называется система частиц \cite{particles1}. Т.~к. с физической точки зрения облако и есть взвешенные в атмосфере частицы воды и/или льда, то 
облака, которые моделируются системой частиц ведут себя достаточно реалистично, но не выглядят кинематографично. В зависимости от задачи, можно вводить различные правила взаимодействия частиц.
Такой подход позволяет моделировать затухание и рождение облака. Но для моделирования одного облака необходимо большое количество частиц, что означает высокие вычислительные затраты и как следствие, визуализиация облачного неба в режиме реального времени, является невыполнимой задачей.  

\subsection{Набор вокселей} 

Воксель - элемент объемного изображения, содержащий значение элемента растра в трехмерном пространстве.
Представление облака с помощью вокселей позволяет использовать клеточные автоматы для создания анимаций. При этом необходимо все пространство моделирования разделить на воксели. Воксели
соответствуют ячейкам, используемым в клеточном автомате. Эволюция облака моделируется путем
применения простых правил перехода на каждом временном шаге. Правила
перехода представляют собой формирование, исчезновение и адвекцию облака ветрами.
Такой подход позволяет создавать реалистично выглядящие и ведущие себя облака. Но по-прежнему, создание облачного неба, требует большое количество ресурсов, для хранения и вычисления состояний вокселей, что делает невозможным моделирование неба в режиме реального времени \cite{voxel}.

\subsection{Участвующая среда} 
\label{partmediaref}
Участвующая среда — это термин, используемый для описания объемов, заполненных частицами. В качестве этих частиц могут быть: капли воды, кристаллы льда, пыль, молекулы. В зависимости от своего состава среда будет по-разному взаимодействовать со светом, проходящим через нее и отражающимся от частиц, что обычно называют светорассеянием. Плотность частиц на единицу объема может изменяться в пространстве. Например, капли воды равномерно распределенные по большому объему, с малой плотностью, могут представлять туман, тогда как локальное, плотное распределение частиц в объеме, может представлять облака \cite{pm}. Для моделирования облаков с помощью участвующей среды, необходимо задать плотность капель воды внутри некоторой области. Сам метод основан на физических уравнениях.
Такой подход позволяет создавать кинематографично выглядящие облака, без больших вычислительных затрат.  


\section{Существующие примеры моделирования объемных облаков}

Использование метода участвующей среды, описанного в пункте \ref{partmediaref}, считается современным стандартом моделирования облаков в компьютерных играх и фильмах. Этот метод применялся в:
\begin{itemize}
	\item Компьютерной игре <<Horizon Zero Dawn>> \cite{hzd};
	\item Игровой движок Frostbite \cite{frostbite};
	\item Фильм <<Оз: Великий и Ужасный>> \cite{oz};
\end{itemize}


\img{75mm}{hzdimg}{Облака созданные с помощью участвующей среды в игре <<Horizon Zero Dawn>>}
\img{75mm}{ozimg}{Облака созданные с помощью участвующей среды и аппроксимацией множественного рассеивания в фильме <<Оз: Великий и Ужасный>>}

\newpage

\section{Модель облака}

\subsection{Прохождение луча через облако}
Известно, что облака представляют собой взвесь частиц воды и льда в воздухе.
Частицы, находящиеся в облаке, проводят и отражают свет, поэтому облако
сильно отличается от твердых объектов и является поглощающей средой.
Закон Бугера — Ламберта — Бера – определяет ослабление пучка света при распространении его в поглощающей среде. 
Свет, проходя сквозь вещество, подвергается поглощению этим самым веществом. Для дифференциального расстояния $ dx $ относительное уменьшение яркости определяется как $ k_a(x)dx $, где $ k_a(x) $ - коэффициент поглощения среды в точке x. Часть света под воздействием частиц вещества меняет свое направление. доля рассеянного света определяется как $ k_s(x)dx $, где $ k_s(x) $ - коэффициент рассеивания. Тогда введем коэффициент затухания $ k_t(x)  = k_a(x) + k_s(x) $, тогда снижение яркости запишем следующим образом (Закон Бера):  
\begin{equation}
\label{for:beers_law}
L(x) = L(x_0) e^{-\int_{x_0}^{x} k_t(u)\,du}=L(x_0)\tau(x_0, x) 
\end{equation}
где $ L(x) $ - яркость в точке $ x $. Функцию $ \tau(x_0, x) = e^{-\int_{x_0}^{x} k_t(u)\,du}  $ называют передаточной функцией, она показывает какая доля света останется при прохождении из точки $ x_0 $ в точку $ x $.




