\chapter{Аналитическая часть}

\section{Участвующая среда}
\label{math}
Участвующая среда -- термин используемый для описания объемов заполненных частицами. Этими частицами могут являться: капли воды, кристаллы льда, пыль, молекулы. Проходя через такую среду, свет взаимодействует с частицами и может отражаться, поглощаться или рассеиваться \cite{frostbite}.  

\subsection{Закон Бугера — Ламберта — Бера}
Закон Бугера — Ламберта — Бера – определяет ослабление пучка света при распространении его в поглощающей среде. 
Свет, проходя сквозь вещество, подвергается поглощению этим самым веществом. Для дифференциального расстояния $ dx $ относительное уменьшение яркости определяется как $ k_a(x)dx $, где $ k_a(x) $ - коэффициент поглощения среды в точке x. Часть света под воздействием частиц вещества меняет свое направление. Доля рассеянного света определяется как $ k_s(x)dx $, где $ k_s(x) $ - коэффициент рассеивания. Тогда введем коэффициент затухания $ k_t(x)  = k_a(x) + k_s(x) $, яркость в точке $x$ находится следующим образом (Закон Бугера — Ламберта — Бера):  
\begin{equation}
	\label{beers_law}
	L(x) = L(x_0) e^{-\int_{x_0}^{x} k_t(u)\,du}=L(x_0)\tau(x_0, x) 
\end{equation}
где $ L(x) $ - яркость в точке $ x $. Функцию $ \tau(x_0, x) = e^{-\int_{x_0}^{x} k_t(u)\,du}  $ называют передаточной функцией, она показывает какая доля света останется при прохождении из точки $ x_0 $ в точку $ x $ \cite{partmedia}.

\subsection{Фазовая функция}
Чтобы вывести уравнение переноса (англ. transport equation), необходимо учитывать увеличение яркости из-за эмиссии и рассеивания света в направлении распространения луча. Пространственное распределение рассеянного света моделируется фазовой функцией $ p(\vec{\omega}, \vec{\omega}') $. Фазовая функция имеет физическую интерпретацию как интенсивность рассеяния в направлении $\vec{\omega}$, деленная на интенсивность, которая была бы рассеяна в этом направлении, если бы рассеяние было изотропным (т.е. независимым от направления). Фазовые функции в компьютерной графике обычно симметричны относительно направления падения, поэтому их можно параметризовать углом между входящим и исходящим лучами. В качестве фазовой функции для облаков используется функция Хеньи — Гринштейна \eqref{HG}:
\begin{equation}
	\label{HG}
	p(\theta) = \frac{1}{4\pi} \frac{1 - g^2}{(1 + g^2 - 2g\cos \theta)^{3/2}}
\end{equation}
где $ g $ - варьируемый параметр, причем $ -1 \leq g \leq 1 $, а $\theta$ - угол между входящим и исходящим лучами \cite{clouds}.

\subsection{Уравнение переноса}
Уравнение переноса описывает изменение яркости внутри участвующей среды в точке $ x $:
\begin{equation}
	\label{rte}
	\begin{aligned}
		\frac{dL(x, \vec{\omega})}{dx} = & k_t(x) J(x, \vec{\omega}) - k_t(x) L(x, \vec{\omega})  = \\
		 & k_a(x) L_e(x, \vec{\omega}) + \frac{k_s(x)}{4\pi} \int_{4\pi} p(\vec{\omega}', \vec{\omega}) L(x, \vec{\omega}') d\omega' - \\
		& k_a(x) L(x, \vec{\omega}) - k_s(x) L(x, \vec{\omega})	
	\end{aligned}
\end{equation}
где $ L_e(x, \vec{\omega}) $ - яркость выделенного света (эмиссии), а $ J(x, \vec{\omega}) $ - вносимая яркость. Стоит заметить, что явление эмиссии не относится к облакам, т.~к. в облаках не наблюдается люминесценция.    
 
Используя альбедо однократного рассеяния $ \Omega = \frac{k_s}{k_t} $, мы можем записать вносимую яркость $J$, как:
\begin{equation}
	\label{sourcerad}
	\begin{aligned}
		J(x, \vec{\omega}) = \underbrace{(1 - \Omega(x)) L_e(x, \vec{\omega})}_{J_e(x, \vec{\omega})} + \frac{\Omega(x)}{4\pi} \int_{4\pi} p(\vec{\omega}', \vec{\omega}) L(x, \vec{\omega}') d\omega'
	\end{aligned}
\end{equation}
Вносимая яркость описывает вклады яркости в луч $ (x, \vec{\omega}) $ в точке $ x $ внутри среды, которые обуславливаются эмиссией света и рассеянными в рассматриваемом направлении лучами. 

Проинтегрировав уравнение переноса \eqref{rte}, получим:
\begin{equation}
	\label{rteint}
	\begin{aligned}
		L(x, \vec{\omega}) =
		\underbrace{\tau(x_0, x) L(x_0, \vec{\omega})}_{L_{ri}(x, \vec{\omega})} + \underbrace{\int_{x_0}^{x} \tau(u, x) k_t(u) J(u, \vec{\omega}) du}_{L_m(x, \vec{\omega})}
	\end{aligned}
\end{equation}
где $ L_{ri}(x, \vec{\omega}) $ - яркость падающего света, а $ L_m(x, \vec{\omega}) $ - яркость, обусловленная явлениями внутри среды. 

Целью алгоритмов рендеринга является разрешение уравнения \eqref{rteint}, по крайней мере для точек видимых относительно камеры \cite{partmedia}. 

\subsection{Однократное рассеивание}

Сложность решения \ref{rteint} заключается в том, что $L$ появляется в обеих частях уравнения (неявно через $J$). Приближенное решение состоит в том, чтобы учитывать только определенное количество событий рассеяния и рассчитывать затухание луча на промежутках между этими событиями. Считаем вклад яркости за счет явлений в среде равным нулю, тогда:
\begin{equation}
	\label{jss}
	\begin{aligned}
		J(x, \vec{\omega}) \approx J_{SS}(x, \vec{\omega}) = J_e(x, \vec{\omega}) + \frac{\Omega(x)}{4\pi} \int_{4\pi} p(\vec{\omega}', \vec{\omega}) L_{ri}(x, \vec{\omega}') d\omega'
	\end{aligned}
\end{equation}

C учетом \ref{jss} и \ref{rteint}, получаем:
\begin{equation}
	\label{rteintss}
	\begin{aligned}
		L(x, \vec{\omega}) =
		\tau(x_0, x) L(x_0, \vec{\omega}) + \int_{x_0}^{x} \tau(u, x) k_t(u) J_{SS}(u, \vec{\omega}) du
	\end{aligned}
\end{equation}

\subsection{Характеристики участвующей среды}
Таким образом, модель участвующей среды, характеризуется: плотностью частиц, коэффициентом поглощения $k_a$, коэффициентом рассеивания $k_s$ и фазовой функцией \cite{clouds}.

В отличие от поверхностей, чьи геометрические и оптические свойства можно рассматривать отдельно, определение геометрии и оптических свойств участвующих сред тесно связаны. Если коэффициент затухания или плотность частиц среды даны непосредственно как функция от координат, то эта функция определяет и геометрию \cite{partmedia}.  


\section{Способы представления облаков}
Методы представления должны определять пространственное распределение плотности, форму и поверхность облаков на сцене.

Для представления облаков существуют следующие методы:
\begin{itemize}
	\item Система частиц;
	\item Иерархическое разделение пространства;
	\item Неявное представление;
	\item Объемы, ограниченные поверхностью (англ. Surface-Bounded Volumes);
\end{itemize}

\subsection{Система частиц}
\label{particles}
Система частиц - используемый в компьютерной графике способ представления объектов, не имеющих четких геометрических границ \cite{particles1}. 
Облако представляется набором частиц с заранее заданным объемом. Обычно используются сферические частицы с радиальной функцией плотности. Очевидно, что если облако будет состоять из большого количества частиц малого объема, то это приведет к огромным вычислительным затратам. Можно оптимизировать этот метод используя метасферы (от англ. meta-balls). Метасфера определяется своим центром, радиусом и плотностью в центре. Значение плотности в любой точке пространства определяется с помощью функции распределения плотностей от расстояния до центра метасферы \cite{nishita}. 

Система частиц представляет собой легкое объемное представление облаков и напрямую поддерживает многие методы облачного моделирования. Сферические частицы хорошо подходят для кучевых или плотных облаков, но менее уместны для слоистых, тонких и перистых облаков.

\subsection{Иерархическое разделение пространства} 

Иерархическое разделение пространства создается над регулярной сеткой. При этом существуют следующие способы построения иерархии:
\begin{itemize}
	\item Воксельные октодеревья. Каждый узел в дереве октантов делит пространство на восемь новых октантов. При этом можно хранить только внешние воксели, которые образуют поверхность облака \cite{clouds};
	\item Двоичное разбиение пространства (ДРП). ДРП, например, в виде kd-деревьев, рекурсивно делит пространство пополам, концентрируясь на областях с высокой геометрической детализацией и удаляя пустые области \cite{clouds};
	\item BVH (англ. Bounding Volume Hierarchies). BVH окружает геометрические объекты сцены оболочками и выстраивает иерархию в виде древовидной структуры данных. Такой подход можно использовать для структурирования системы частиц, описанной в \ref{particles};
\end{itemize}

 В общем случае эти методы позволяют представлять все типы облаков. Но в силу того, что они используют регулярную сетку, их прямое использование неэффективно для сцен большого масштаба, ведь тогда регулярная сетка занимает большой объем памяти, причем большинство ячеек сетки могут быть пустыми. Чаще всего в этом случае, сетку вкладывают в сферическую оболочку или используют трассировку лучей, которая учитывает искривление поверхности Земли, таким образом получается существенно снизить затраты по памяти, т.~к. снижается размер видимой части сетки без вреда для восприятия сцены (облака плавно скрываются за горизонтом) \cite{clouds}.


\subsection{Неявное представление} 
\label{implicit}
Распространенным способом моделирования и представления поля плотности облака является использование процедурных методов. В то время как общая форма обычно задается простыми геометрическими объектами, такими как сферы или эллипсоиды, внутренняя структура с высоким разрешением моделируется процедурно \cite{clouds}. Существуют различные способы неявного представления облаков. Например, в \cite{implicit} использовалась комбинация полигональной сетки и процедурно сгенерированных текстур. Сетка применяется для описания внешней границы облака в низком разрешении. Процедурно сгенерированные, 3D-текстуры добавляют деталей к этой границе, при этом среда внутри сетки считается однородной.

Другой подход заключается в использовании двух 3D-текстур, одна из которых состоит из низкочастотного шума и служит для формирования общей формы облака, а другая используется для создания мелких деталей на поверхности. Сами текстуры состоят из шумов Перлина и Ворлея с разными частотами и амплитудами. Форма и положение облака задаются с помощью погодной карты и функции разрежения поля плотности в зависимости от высоты. Также форма и положение облака могут задаваться аналитически с помощью геометрических поверхностей \cite{frostbite, hzd}.

\subsection{Объемы, ограниченные поверхностью} 
Облачный объем представляется окружающей его оболочкой, обычно заданной с помощью полигональной сетки. Поскольку информация о внутренней структуре отсутствует, обычно предполагается однородный объем. Такой подход является компактным и быстрым способом представления облаков, однако он применим только в том случае, если можно предположить быстрое насыщение луча, входящего в объем. Т.~о. этот метод применим для плотных облаков, но не подходит для представления тонких или слоистых \cite{clouds}. 

\subsection{Выбор способа представления}

В последнее время, крупные компании \cite{frostbite, hzd}, выпускающие игры, используют для представления облаков процедурную генерацию, основанную на двух 3D-текстурах, описанную в пункте \ref{implicit}. Такой выбор обусловлен огромной экономией памяти по сравнению с другими методами. Использование памяти для визуализации всего неба ограничено стоимостью двух 3D-текстур и одной 2D-текстуры (погодной карты), т.~е. составляет примерно 20 МБ \cite{hzdres}. В то же время для хранения облачного неба в виде системы частиц или с помощью иерархии на регулярной сетке, требуется в десятки раз больше памяти.

 
\section{Анализ методов визуализации облаков}

Рассмотрим следующие методы рендеринга облаков:

\begin{itemize}
	\item Срез объема (англ. Volume slicing);
	\item Разбрызгивание текстур (англ. Splatting);
	\item Ray marching;
\end{itemize}


\subsection{Срез объема}
\label{slice}
Срез объемов - это простой метод рендеринга регулярных сеток. Срезы объема производятся перпендикулярно каждой из главных осей или линии взгляда, и результирующая информация для каждого среза представляется в виде 2D-текстуры. Рендеринг выполняется путем проецирования текстур и их смешивания в буфере кадра \cite{vs}. Нарезка объема плоскостями, ориентированными под углом, равным половине угла между направлением освещения и направлением взгляда, называется нарезкой по половинному углу (англ. Half-Angle Slicing). Такой подход позволяет совместить освещение и визуализацию объема в одном процессе путем однократной итерации по всем срезам. Во время этого единственного прохода объема поддерживаются и итеративно обновляются два буфера: один для накопления ослабления яркости в направлении распространения света, а другой для накопления яркости для наблюдателя. Из-за однократного прохождения через объем схема освещения ограничивается либо прямым, либо обратным рассеянием \cite{clouds}. 

Чтобы избежать появления артефактов, рекомендуется использовать множество срезов с малым шагом, что может привести к снижению производительности рендеринга. Передача сложной геометрии объема также может потребовать большого количества срезов. Следовательно, методы, основанные на срезах, предпочтительны для объемов с мягкими или размытыми границами, однако их применимость ограничена в случае, когда объем имеет резкие границы \cite{clouds}.

\subsection{Разбрызгивание текстур}
\label{splatting}
Разбрызгивание стало распространенным методом рендеринга систем частиц. Частицы, которые обычно определяются как независимые от вращения, могут быть визуализированы с помощью текстурированного четырехугольника, представляющего проекцию частицы на плоскость, также называемую <<пятном>> или <<отпечатком>> (от англ. splat и footprint соответственно). Частицы визуализируются в обратном порядке, применяя смешивание текстур в буфере кадра \cite{voxel, clouds}. 

Частицы представляют некоторый сферический, рассеивающий объем, а не четкий геометрический объект, поэтому этот метод подходит для визуализации облаков с мягкими, пушистыми формами. Облака с четкой геометрией поверхности, такие как кучевые облака, не могут быть воспроизведены реалистично.


\subsection{Ray marching}
\label{ray}
Ray marching бросает лучи в сцену и накапливает объемную плотность через определенные интервалы. Для визуализации облаков, нужно учесть освещение, это можно сделать либо на лету применив модель освещения, либо путем извлечения значений из предварительно вычисленной структуры данных освещения \cite{hzd, frostbite, clouds}.

Ray marching позволяет достичь более реалистичного рендеринга облаков, поскольку он учитывает сложные внутренние структуры объема и свойства облака. Он также способен обрабатывать более сложные геометрические формы облаков, такие как кучевые облака.

\subsection{Выбор метода визуализации}
Для визуализации неявно представленных облаков применяется метод ray marching, поскольку поле плотности в таких облаках задается текстурой, а не системой частиц или регулярной сеткой. Трассировку лучей можно выполнять параллельно, за счет когерентности пикселей, чем и пользуются в современных играх \cite{hzd, frostbite}. Используя вычисления на GPU получается добиться того, что алгоритм ray marching, оказывается быстрее всех описанных выше алгоритмов и при этом, позволяет визуализировать облака любой формы.


\section{Модель освещения}

Для реалистичной визуализации облаков, модель освещения должна аппроксимировать следующие эффекты: множественное рассеяние и направленное освещение в облаках, <<серебряное обрамление>> (эффект возникающий, когда мы смотрим на солнце через облака), и темные границы облака, когда мы смотрим на облако в направлении от солнца. 

Как было описано в пункте \ref{math}, с фотоном попавшим в облако, может произойти три события:
\begin{enumerate}
	\item Он может поглотиться частицей воды или другой частицей, не относящейся к облаку, например пылью;
	\item В ходе отражений, он выходит из облака и движется в сторону наблюдателя. Назовем это внутренним рассеянием;
	\item В ходе отражений, он выходит из облака и движется в сторону противоположную от наблюдателя. Назовем это внешним рассеянием;
\end{enumerate}
Закон Бера \eqref{beers_law} является стандартным методом для аппроксимации вероятности каждого из этих трех явлений. Т.~к. облака являются неоднородной средой, мы должны накапливать оптическую плотность вдоль луча. Тогда, приближенно, считаем энергию накопленную лучом:
\begin{equation}
	\label{bl}
	BL = e ^ {-d}
\end{equation}
где $ d $ - накопленная оптическая плотность.


Для моделирования явления <<серебряного обрамления>>, используются фазовые функции, одной из которых, является функция Хеньи-Гринштейна \eqref{HG}.

Закон Бера — это модель поглощения, а это означает, что он описывает то, как световая энергия поглощается при распространении вглубь облака. Но такая модель не учитывает важный световой эффект, который связан с внутренним рассеянием на обращенных к солнцу сторонах облаков. Этот эффект проявляется в виде темных краев на облаках, когда направление луча обзора приближается к направлению луча света. Этот эффект наиболее заметен в круглых, плотных областях облаков, настолько, что складки между каждой выпуклостью кажутся ярче, чем сама выпуклость, которая находится ближе к солнцу. Эти результаты кажутся полной противоположностью тому, что моделирует закон Бера.  Напомним, что большее количество света рассеивается вперед вдоль исходного направления светового луча из-за прямого рассеяния. Должна существовать относительно большая оптическая толщина, чтобы фотон мог повернуться на 180 градусов. Пути около границы облака не проходят через достаточно большую оптическую толщину, чтобы полностью обернуть заметную часть фотонов. Пути, которые имеют оптическую толщину, достаточную для поворота фотона на 180 градусов, почти всегда находятся внутри облака, поэтому закон Бера заглушит этот вклад до того, как он покинет облако к нашему глазу. Щели и трещины являются исключением, они создают <<окно>>, через которое фотоны могут покинуть облако по пути с малой плотностью, делая расщелины ярче, чем окружающие их выпуклости. Для учета этого эффекта в закон Бера вносят поправку:

\begin{equation}
	\label{sugar_powder}
	PSE = 1 - e ^ {-2 d}
\end{equation}

Приближено, считаем световую энергию, накопленную лучом, таким образом:
\begin{equation}
	\label{bsp}
	LE = 2 * BL * PSE
\end{equation}
где $ LE $ - накопленная энергия света. 

В итоге, полностью наша модель освещения описывается следующим образом:
\begin{equation}
	\label{light_model}
	E = 2 * e ^ {-d} * (1 - 2 e ^ {-2d}) * \frac{1}{4\pi} \frac{1 - g^2}{(1 + g^2 - 2g\cos \theta)^{3/2}}
\end{equation}
где $ g $ - некоторая константа в функции Хеньи-Гринштейна, а $ \theta $ - угол между лучами взгляда и света.

\subsection{Конусное освещение (англ. Cone-Sampled Lighting)}
Очевидный способ найти количество падающего света — измерить переносимую световую энергию внутри облака между рассматриваемой точкой и солнцем. Однако на освещенность любой точки облака сильно влияет освещенность в областях вокруг нее в направлении источника света, т.~е. нам необходимо рассмотреть конус с вершиной в рассматриваемой точке. Внутри конуса мы выбираем случайным образом, некоторое количество точек, в которых рассчитываем плотность. Суммируя плотности в точках, мы получаем плотность накопленную в конусе, далее используем ее, чтобы вычислить световую энергию в рассматриваемой точке.


%%% --------------------------------- COMMENTED -----------------------------
\mycomment{
\section{Техники освещения}

\subsection{Аппроксимация однократного рассеивания}
\begin{itemize}
	\item На основе срезов: нарезанный объем используется для создания объемной карты освещения, хранящей яркость света. Этот так называемый <<световой объем>> ориентирован так, что свет проходит перпендикулярно или параллельно срезам из которых он состоит, что обеспечивает прямое распространение света от среза к срезу \cite{schpok};
	\item На основе частиц. Используется отбрасывания теней в качестве аппроксимации однократного рассеяния для освещения системы частиц. Сцена преобразуется таким образом, чтобы точка наблюдения совпала с источником света. Кадровый буфер графического процессора используется в качестве карты теней. Частицы сортируются и обрабатываются в направлении спереди назад относительно источника света. Для каждой частицы значение тени рассчитывается с помощью коэффициента затухания, который в свою очередь рассчитывается итеративно, т.~е. коэффициент затухания в некоторой частице равен произведению коэффициентов всех частиц, через который прошел свет на пути к данной частице \cite{voxel};
\end{itemize}

\subsection{Диффузия света}

 В оптически плотных средах, когда свет сталкивается с множеством частиц, диффузия является допустимым приближением для описания распределения света внутри таких сред. Однако это приближение становится менее точным в неоднородных средах или на их границах. 
 Возможно комбинировать диффузию с анизотропным рассеиванием, чтобы учитывать неоднородности в среде, например, чтобы воспроизвести такой эффект, как <<серебряное обрамление>> (англ. silver-lining) в облаках. <<Серебряное обрамление>> - это светящаяся граница облаков, которая может наблюдаться при правильных условиях освещения.
 Таким образом, данная модель может быть эффективной для создания более реалистичных визуальных эффектов в облаках, но требует значительных вычислительных ресурсов и рендеринг может занимать несколько часов \cite{clouds}.


\subsection{Трассировка пути}

Подход заключается в том, что для каждого пикселя на изображении запускаются сотни или даже тысячи лучей в сцену, которые отслеживаются до тех пор, пока они не достигнут источника света или не будут отражены обратно в камеру. Метод <<Трассировки пути>> обеспечивает физически корректные результаты, но, как правило, страдает от шума и низкой скорости работы.

Различные структуры данных используются для разделения объемных данных на подмножества для эффективного обхода и отслеживания лучей. В качестве таких структур данных могут выступать kd-деревья или регулярные сетки. Такие ускоренные методы позволяют сократить время рендеринга и снизить уровень шума, делая метод <<Трассировки пути>> более применимым для сложных объемных данных, таких как облака, и позволяют получить более качественные и реалистичные результаты \cite{clouds}.

\subsection{Фотонные карты}

Фотоны в данном методе — это частицы, переносящие некоторую дискретную порцию световой энергии. На начальном этапе фотоны испускаются из источника света в соответствии с распределением световой энергии у данного источника. Например известно, что точечный или сферический источник света испускают свет изотропно во всех направлениях. В  процессе  трассировки  фотоны  ударяются  о  различные  поверхности  сцены.  В 
зависимости от свойств материала поверхности, с ними могут происходить разные события: 
прохождение через поверхность, полное поглощение, зеркальное отражение либо диффузное отражение. При  взаимодействии  фотона  с  диффузным  объектом  запись  о  фотоне 
сохраняется в обычном списке или массиве. 

После  завершения  трассировки  фотонов  необходимо  произвести  непосредственно 
построение  фотонной  карты,  которая  представляет  собой  специальную  структуру 
распределения  фотонов  на  поверхностях  сцены.  Данный  этап  позволяет  избежать 
значительных вычислительных затрат, поскольку если бы фотоны просто сохранялись в виде 
массива  или  списка,  то  k  ближайших  фотонов  к  текущему  каждый  раз  приходилось  бы 
вычислять перебором, что значительно снижает быстродействие алгоритма. Для хранения фотонов используется kd-дерево.

Далее применяется обычная трассировка лучей из камеры, но в местах пересечения лучей с объектами находятся k ближайших фотонов, их энергия суммируется и делится на площадь сферы с радиусом равным расстоянию да самого дальнего из этих фотонов. Полученная величина добавляется к общей энергии собранной лучом. Важным  достоинством  метода  фотонных  карт  является  тот  факт,  что  расчеты  не 
зависят  от  положения  камеры,  что  позволяет во  многих  случаях  рассчитать  освещенность 
всего  один  раз \cite{photon}.

\subsection{Глобальная модель освещения}

Для реализации явлений межоблачного затенения и непрямого освещения используется глобальная модель освещения.

Представление объектов сцены с помощью сфер, позволяет быстро вычислять тени, например, используя сферические функции, или путем накопления теней в пространстве изображения \cite{clouds}.

Непрямое освещение: освещенность точки на поверхности оценивается с помощью освещенности соседних точек в нескольких направлениях, ограниченных конусами.
}
\section{Виды шумов, применяемых в процедурной генерации облаков}

\subsection{Шум Ворлея}

Для генерации шума Ворлея необходимо создать регулярную сетку некоторого размера и разместить в центре каждой ячейки точку. Далее используя хеш-функцию, мы сдвигаем каждую точку в пределах ячейки. Теперь, чтобы рассчитать шум для любого пикселя, нужно определить в какую ячейку он попадает и какие ячейки ему смежны. Далее проходя по всем ячейкам, мы определяем расстояние между нашим пикселем и смещенными точками, значение шума Вороного – минимальное из расстояний, а инвертировав шум Вороного, получаем искомый шум Ворлея. 
Недостаток – этот шум не складывается воедино, т.~е. при сложении текстур созданных с помощью шума, получаются резкие стыки, чтобы это исправить, необходимо брать смещение точки по модулю \cite{worley}.

\subsection{Шум Перлина}

Шум Перлина — это градиентный шум, состоящий из набора псевдослучайных единичных векторов (направлений градиента), расположенных в определенных точках пространства и интерполированных функцией сглаживания между этими точками. Беря по модулю псевдослучайные числа, добиваемся того, что текстуры созданные с помощью шума, складываются воедино. 



\section{Существующие примеры моделирования объемных облаков}
	
	Использование процедурной генерации описанной в пункте \ref{implicit}, считается современным стандартом моделирования облаков в компьютерных играх и фильмах. Этот метод применялся в:
	\begin{itemize}
		\item Компьютерной игре <<Horizon Zero Dawn>> \cite{hzd};
		\item Игровом движке Frostbite \cite{frostbite};
	\end{itemize}
	
	
	\img{75mm}{hzdimg}{Облака созданные с помощью процедурной генерации и погодных карт в игре <<Horizon Zero Dawn>>}








