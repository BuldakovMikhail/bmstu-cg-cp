\chapter{Аналитическая часть}

\section{Участвующая среда}

Участвующая среда - термин используемый для описания объемов заполненных частицами. Этими частицами могут являться: капли воды, кристаллы льда, пыль, молекулы. Проходя через такую среду, свет взаимодействует с частицами и может отражаться, поглощаться или рассеиваться \cite{frostbite}.  

\subsection{Прохождение луча через облако}
Известно, что облака представляют собой взвесь частиц воды и льда в воздухе.
Частицы, находящиеся в облаке, проводят и отражают свет, поэтому облако
сильно отличается от твердых объектов и является поглощающей средой.
Закон Бугера — Ламберта — Бера – определяет ослабление пучка света при распространении его в поглощающей среде. 
Свет, проходя сквозь вещество, подвергается поглощению этим самым веществом. Для дифференциального расстояния $ dx $ относительное уменьшение яркости определяется как $ k_a(x)dx $, где $ k_a(x) $ - коэффициент поглощения среды в точке x. Часть света под воздействием частиц вещества меняет свое направление. Доля рассеянного света определяется как $ k_s(x)dx $, где $ k_s(x) $ - коэффициент рассеивания. Тогда введем коэффициент затухания $ k_t(x)  = k_a(x) + k_s(x) $, яркость в точке $x$ находится следующим образом (Закон Бера):  
\begin{equation}
	\label{for:beers_law}
	L(x) = L(x_0) e^{-\int_{x_0}^{x} k_t(u)\,du}=L(x_0)\tau(x_0, x) 
\end{equation}
где $ L(x) $ - яркость в точке $ x $. Функцию $ \tau(x_0, x) = e^{-\int_{x_0}^{x} k_t(u)\,du}  $ называют передаточной функцией, она показывает какая доля света останется при прохождении из точки $ x_0 $ в точку $ x $.


\section{Способы представления облаков}

Представить объемные облака можно следующими способами:
\begin{itemize}
	\item Система частиц;
	\item Набор вокселей;
	\item Объемы, ограниченные поверхностью (англ. Surface-Bounded Volumes);
\end{itemize}

\subsection{Система частиц}
\label{particles}
Система частиц - используемый в компьютерной графике способ представления объектов, не имеющих четких геометрических границ \cite{particles1}. 
Облако представляется набором частиц с заранее заданным объемом. Обычно используются сферические частицы с радиальной функцией плотности. Т.~к. с физической точки зрения облако и есть взвешенные в атмосфере частицы воды и/или льда, то 
облака, которые моделируются системой частиц ведут себя достаточно реалистично. В зависимости от задачи, можно вводить различные правила взаимодействия частиц.
Такой подход позволяет моделировать затухание и рождение облака. Но для моделирования одного облака необходимо большое количество частиц, что означает высокие вычислительные затраты и как следствие, визуализиация облачного неба в режиме реального времени, является сложной задачей.  

\subsection{Набор вокселей} 

Воксель - элемент объемного изображения, содержащий значение элемента растра в трехмерном пространстве.
Представление облака с помощью вокселей позволяет использовать клеточные автоматы для создания анимаций. При этом необходимо все пространство моделирования разделить на воксели. Воксели
соответствуют ячейкам, используемым в клеточном автомате и содержат информацию о своих свойствах, таких как плотность, цвет и освещение. Эволюция облака моделируется путем
применения простых правил перехода на каждом временном шаге. Правила перехода представляют собой формирование, исчезновение и адвекцию облака ветрами.
Такой подход позволяет создавать реалистично выглядящие и ведущие себя облака. Но по-прежнему, создание облачного неба, требует большое количество ресурсов, для хранения и вычисления состояний вокселей, что делает сложным моделирование неба в режиме реального времени \cite{voxel}.

\subsection{Объемы, ограниченные поверхностью} 
\label{partmediaref}
Распространенным способом моделирования и представления облаков является использование процедурных методов. Облачный объем представляется окружающей его оболочкой, обычно заданной с помощью полигонов. Использование оболочки комбинируется с использованием шумовых функций для добавления мелких деталей на поверхности облака. Внутреннее распределение плотности может храниться в заранее заготовленных картах или вычисляться на лету с помощью шумов \cite{frostbite}. Такой подход позволяет эффективно выполнять визуализацию облаков и использовать сложные системы освещения \cite{clouds}. 


\section{Анализ методов визуализации облаков}

Рассмотрим следующие методы рендеринга облаков:

\begin{itemize}
	\item Срез объема (англ. Volume slicing);
	\item Разбрызгивание текстур (англ. Splatting);
	\item Ray marching;
\end{itemize}


\subsection{Срез объема}
\label{slice}
Срез объемов - это простой метод рендеринга регулярных сеток. Срезы объема производятся перпендикулярно каждой из главных осей или линии взгляда, и результирующая информация для каждого среза представляется в виде 2D-текстуры. Рендеринг выполняется путем проецирования текстур и их смешивания в буфере кадра \cite{vs}. Нарезка объема плоскостями, ориентированными под углом, равным половине угла между направлением освещения и направлением взгляда, называется нарезкой по половинному углу (англ. Half-Angle Slicing). Такой подход позволяет совместить освещение и визуализацию объема в одном процессе путем однократной итерации по всем срезам. Во время этого единственного прохода объема поддерживаются и итеративно обновляются два буфера: один для накопления ослабления яркости в направлении света, а другой для накопления яркости для наблюдателя. Из-за однократного прохождения через объем схема освещения ограничивается либо прямым, либо обратным рассеянием \cite{clouds}. 

Чтобы избежать появления артефактов, рекомендуется использовать множество срезов с малым шагом, что может привести к снижению производительности рендеринга. Передача сложной геометрии объема также может потребовать большого количества срезов. Следовательно, методы, основанные на срезах, предпочтительны для объемов с мягкими или размытыми границами, однако их применимость ограничена в случае, когда объем имеет резкие границы.

\subsection{Разбрызгивание текстур}
\label{splatting}
Разбрызгивание стало распространенным методом рендеринга систем частиц. Частицы, которые обычно определяются как независимые от вращения, могут быть визуализированы с помощью текстурированного четырехугольника, представляющего проекцию частицы на плоскость, также называемую <<пятном>> или <<отпечатком>> (от англ. splat и footprint соответственно). Частицы визуализируются в обратном порядке, применяя смешивание текстур в буфере кадра \cite{voxel, clouds}. 

Частицы представляют некоторый сферический, рассеивающий объем, а не четкий геометрический объект, использование метода <<разбрызгивания текстур>> ограничено.  Этот метод подходит для визуализации облаков с мягкими, пушистыми формами. Облака с четкой геометрией поверхности, такие как кучевые облака, не могут быть воспроизведены реалистично.


\subsection{Ray marching}
\label{ray}
Ray marching бросает лучи в сцену и накапливает объемную плотность через определенные интервалы. Для визуализации облаков, нужно учесть освещение, это можно сделать либо на лету применив модель освещения, либо путем извлечения значений из предварительно вычисленной структуры данных освещения \cite{hzd, frostbite, clouds}.

Ray marching позволяет достичь более реалистичного рендеринга облаков, поскольку он учитывает сложные внутренние структуры объема и свойства облака. Он также способен обрабатывать более сложные геометрические формы облаков, такие как Cumulus облака.


\section{Глобальная модель освещения}




\section{Существующие примеры моделирования объемных облаков}

Использование процедурной генерации описанной в пункте \ref{partmediaref}, считается современным стандартом моделирования облаков в компьютерных играх и фильмах. Этот метод применялся в:
\begin{itemize}
	\item Компьютерной игре <<Horizon Zero Dawn>> \cite{hzd};
	\item Игровом движке Frostbite \cite{frostbite};
	\item Фильме <<Оз: Великий и Ужасный>> \cite{oz};
\end{itemize}


\img{75mm}{hzdimg}{Облака созданные с помощью процедурной генерации и погодных карт в игре <<Horizon Zero Dawn>>}
\img{75mm}{ozimg}{Облака созданные с помощью участвующей среды и аппроксимацией множественного рассеивания в фильме <<Оз: Великий и Ужасный>>}








